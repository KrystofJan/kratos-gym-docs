% Nejprve uvedeme tridu dokumentu s volbami
\documentclass[czech,bachelor]{diploma}
% Dalsi doplnujici baliky maker
\usepackage[autostyle=true,czech=quotes]{csquotes} % korektni sazba uvozovek, podpora pro balik biblatex
\usepackage[backend=biber, style=iso-numeric, alldates=iso]{biblatex} % bibliografie
\usepackage{dcolumn} % sloupce tabulky s ciselnymi hodnotami
\usepackage{subfig} % makra pro "podobrazky" a "podtabulky"
\usepackage[cpp]{diplomalst}
\usepackage{dirtree}

% Zadame pozadovane vstupy pro generovani titulnich stran.
\ThesisAuthor{Jan-Kryštof Zahradník}

\ThesisSupervisor{Ing. Lukáš Klein}

\CzechThesisTitle{Navržení a implementace rezervačního systému pro posilovny s ohledem na maximální kapacitu, typ tréninku a využití nástrojů}

\EnglishThesisTitle{Design and Implementation of a Booking System for Gyms with Regard to Maximum Capacity, Type of Training and Use of Tools}

\SubmissionYear{2024-2025}

\ThesisAssignmentFileName{ThesisSpecification_ZAH0089.pdf}

% Pokud nechceme nikomu dekovat makro zapoznamkujeme.
\Acknowledgement{Rád bych zde poděkoval vedoucímu mé práce Ing. Lukáši Kleinovi za pomoc a cenné rady při tvorbě této bakalářské práce}

\CzechAbstract{Tato bakalářská práce se zaměřuje na návrh a implementaci rezervačního systému pro posilovny s důrazem na správu kapacity, typů tréninků a dostupných zdrojů. Výsledkem je webová aplikace, která uživatelům umožňuje rezervovat jednotlivé posilovací stroje v konkrétním čase, navrhovat vhodné stroje odpovídající jejich tréninkovým plánům a optimalizovat čas tréninku na základě zvoleného počátečního času.}

\CzechKeywords{typografie; \LaTeX; diplomová práce}

\EnglishAbstract{This bachelor's thesis focuses on the design and implementation of a reservation system for gyms, emphasizing efficient management of capacity, training types, and available resources. The result is a web application that allows users to book individual exercise machines at specific times, suggest suitable machines that align with their training plans, and optimize training schedules based on the selected start time.}

\EnglishKeywords{typography; \LaTeX; master thesis}

\AddAcronym{API}{Application Programming Interface}
\AddAcronym{REST API}{Representational State Transfer API}
\AddAcronym{JS}{JavaScript}
\AddAcronym{TS}{TypeScript}
\AddAcronym{PK}{Primární klíč}
\AddAcronym{FK}{Cizí klíč}

\addbibresource{biblatex-examples.bib}
\addbibresource{coffee.bib}

% Novy druh tabulkoveho sloupce, ve kterem jsou cisla zarovnana podle desetinne carky
\newcolumntype{d}[1]{D{,}{,}{#1}}


% Zacatek dokumentu
\begin{document}

% Nechame vysazet titulni strany.
\MakeTitlePages

% Jsou v praci obrazky? Pokud ano vysazime jejich seznam a odstrankujeme.
% Pokud ne smazeme nasledujici dve makra.
\listoffigures
\clearpage

% Jsou v praci tabulky? Pokud ano vysazime jejich seznam a odstrankujeme.
% Pokud ne smazeme nasledujici dve makra.
\listoftables
\clearpage

% A nasleduje text zaverecne prace.
\chapter{Úvod}
\label{sec:Introduction}
V současné době převažují ve fitness komunitě rezervační systémy odlišné od návrhu představeného v této bakalářské práci. Stávající řešení se primárně zaměřují na rezervaci osobních tréninků s certifikovanými trenéry nebo na alokaci prostoru pro skupinové aktivity, jako jsou spinningové lekce, boxerské tréninky či jiné specializované kolektivní programy. Tyto systémy však nedostatečně řeší problém přetížení fitness center, který se výrazně projevuje zejména v období zvýšené poptávky (např. na začátku roku). Následkem toho jsou návštěvníci nuceni průběžně upravovat své tréninkové plány podle aktuální dostupnosti přístrojů a kapacity zařízení, což snižuje jak efektivitu tréninku, tak celkovou spokojenost klientů.

Hlavní limitace současných systémů tkví v neschopnosti dynamicky reagovat na vytíženost jednotlivých přístrojů. Místo toho, aby monitorovaly stav zařízení v reálném čase (např. obsazenost běžeckých pásů, činek nebo strojů). To vede k nerovnoměrnému vytížení vybavení, častým časovým překryvům a nevyužitým kapacitám v méně frekventovaných časech.

Tento problém je mimořádně komplexní, protože je nutné zohlednit nejen maximální kapacitu posilovny, ale také vyhnout se kolizím v používání jednotlivých přístrojů. Na trhu i v open-source komunitě momentálně neexistuje žádné podobné řešení, které by tento problém řešilo efektivním a škálovatelným způsobem.

Tato bakalářská práce se zaměřuje na návrh a implementaci rezervačního systému pro posilovnu, který inteligentně doporučuje vhodné časy a přístroje pro jednotlivé uživatele. Výsledkem je REST API s webovou aplikací, která umožňuje uživatelům intuitivně spravovat své rezervace a získávat optimální doporučení pro jejich tréninkové plány.
\endinput
\chapter{Popis použitých technologií}  \label{techstack}

Tato kapitola popisuje technologie použité ve vývoji této bakalářské práce a vysvětluje jejich funkci v rámci jednotlivých vrstev. Ústřední roli v celém ekosystému hraje JavaScript, který se stal jádrem celého vývojového procesu. Tato volba vycházela z mnoha později uvedených důvodů, ale i přes to, že odpovídá široce rozšířené praxi, byla klíčová pro volbu ostatních technologií.

\section{JavaScript}
JavaScript je základním stavebním kamenem projektu. Byla v něm i napsána úplně první verze celého backendu. V projektu byl JavaScript použit z několika důvodů:

\begin{description}
  \item[Rozšířená komunita] \hfill \\ Javascript má obrovskou komunitu vývojářů, která nabízí bohatou podporu a nespočet open-source knihoven a frameworků, což usnadňuje a zrychluje vývoj.
  \item[Nativní podpora ve webových prohlížečích] \hfill \\ I v dnešní době, kdy máme mnohem více možností, je javascript jazyk stále "Jazykem webu" a běží nativně ve všech moderních prohlížečích, což z něj činí téměř nezbytnou součást vývoje webových aplikací. Pro jednoduchost a jednotu mezi vrstvami dávalo smysl používat jeden jazyk.
\end{description}

Použití JavaScriptu umožnilo rychlý start a flexibilní rozvoj projektu. Samozřejmě, že jeho volba s sebou přinesla i některé problémy a úskalí, se kterými jsem se musel v průběhu projektu vypořádat. Tyto přešlapy si později rozebereme v retorspektivě.

V pozdější části implementace došlo k rozhodnutí přejít na TypeScript, což bylo důležitým krokem směrem k lepší struktuře a stabilitě kódu.

\subsection{TypeScript}
Je to open-source programovací jazyk vyvíjen společností Microsoft \cite{enwiki:1261773908}, který je postavený ja JavaScriptu a rozšiřuje ho o typovou kontrolu\cite{goldberg2022learning}. TypeScript poskytuje výhodu především ve větších projektech, kde je potřeba zajistit stabilitu a minimalizovat chyby spojené s dynamickým typováním JavaScriptu.

Statická typová kontrola v TypeScriptu umožňuje odhalit chyby již během psaní kódu a nabízí větší jistotu při refactoringu \cite{enwiki:1258410189}. Přestože TypeScript stále generuje běžný JavaScript, přináší strukturálnější a bezpečnější vývojový proces, což významně zlepšilo kvalitu kódu v tomto projektu.

\section{PostgreSQL}
PostgreSQL je open-source objektově-relační databázový systém, který byl v tomto projektu zvolen jako hlavní databázové řešení \cite{enwiki:1262241967}. Tento databázový systém je známý svou stabilitou, rozšiřitelností a podporou pokročilých funkcí, jako jsou uložené procedury, transakce nebo plná podpora ACID (Atomicita, Konzistence, Izolace, Trvalost). PostgreSQL nabízí škálovatelnost a spolehlivost, které jsou klíčové pro náročné webové aplikace.

\section{Node.js}
Aby program běžel, tak většína programovaních jazyků má nějaké běhové prostředí (Ang. Runtime environment) \cite{enwiki:1245152116}. Při vývoje JavaScriptu byl vytvořen "JavaScript engine" pro webový prohlížeč Netscape Navigator. V podstatě se jednalo o velmi jednoduchý interpreter. Později se tento engine se později vyvinul v SpiderMonkey, který je dodnes používán v prohlížeči Mozilla Firefox\cite{enwiki:1256640147}. Alternativa pro chromium se nazývá V8.

Node.js poskytuje běhové prostředí pro JavaScript mimo prohlížeč. Vznikl v roce 2009 a umožnil použití JavaScriptu pro serverové aplikace. Je postaven na enginu V8 a díky tomu je velmi výkonný a efektivní.\cite{enwiki:1262522147}

Node.js je dnes největším běhovým prostředím pro JavaScript na světě. Kromě možnosti tvorby serverových aplikací poskytuje ekosystém balíčků spravovaný přes npm (Node Package Manager), což je největší repozitář softwarových balíčků na světě.\cite{enwiki:1262522147}

Node má samozřejmě spoustu moderních alternativ jako je bun, deno a další, ale považován za standard pro vývoj aplikací v JS/TS. 

\section{Express.js}
Express.js je minimalistický webový framework pro Node.js, který byl použit pro tvorbu aplikačního programového rozhraní (API). Express usnadňuje práci s HTTP metodami a nabízí jednoduchou, ale flexibilní strukturu pro vývoj REST API\cite{enwiki:1258184667}.

Express umožnuje jednoduchou, ale transparentní práci s HTTP metodami z něho tvoří oblíbenou volbu pro vývoj awebových aplikací všech velikostí. V tomto projektu funguje jako komunikační most mezi frontendem a databází.

\section{Vue.js}
Vue.js je progresivní open-source framework pro tvorbu uživatelských rozhraní. Byl zvolen pro frontend tohoto projektu díky své jednoduchosti a flexibilitě. Vue.js 
umožňuje rychlý vývoj komponent, které jsou zabaleny do velmi přehledné struktury.

Hlavní silnou stránkou Vue.js je jeho reaktivní systém správy stavu. Tento mechanismus automaticky sleduje závislosti mezi daty a rozhraním, což eliminuje potřebu explicitního řízení aktualizací.
%Mozna odstranit?%
Zatímco frameworky jako React spoléhají na imperativní manipulaci se stavem pomocí hooků (např. useState), Vue.js umožňuje přímou mutaci reaktivních proměnných deklarovaných pomocí ref() nebo reactive(). Tento rozdíl redukuje boilerplate kód a snižuje riziko chyb spojených s nesprávnou synchronizací stavu. Následující ukázka demonstruje implementaci čítače v Reactu s použitím hooku useState:
\begin{lstlisting}
function Example() {
  const [count, setCount] = useState(0);
  return (
    <div>
      <p>You clicked {count} times</p>
      <button onClick={() => setCount(count + 1)}>
        Click me
      </button>
    </div>
  );
\end{lstlisting}
Zde je nutné explicitně deklarovat jak stavovou proměnnou (count), tak odpovídající setter (setCount). Každá změna stavu vyžaduje volání setter funkce, což vede k přerenderování komponenty. Ačkoli tento model poskytuje granularitu kontroly, vyžaduje od vývojáře důsledné oddělení stavové logiky a prezentace. Naproti tomu stejná funkcionalita ve Vue.js využívá reaktivní referenci:
\begin{lstlisting}
<script setup>
import { ref } from "vue";
const counter = ref(0);
</script>

<template>
  <div>
    <p>You clicked {{ counter }} times</p>
    <button @click="counter++">Click me</button>
  </div>
</template>
\end{lstlisting}
Proměnná counter je obalena proxy objektem, který automaticky detekuje změny prostřednictvím getterů a setterů generovaných funkcí ref(). Mutace hodnoty přímým přiřazením (counter++) spouští reaktivní aktualizaci DOM bez nutnosti manuálního volání setterů.

\section{Tailwind CSS}
Tailwind je framework pro CSS, který výrazně zrychluje stylování uživatelského rozhraní. Namísto tradičních předpřipravených komponent nabízí Tailwind množství utility tříd, které lze kombinovat a vytvářet tak design přímo v HTML komponentách.

Použití Tailwindu v tomto projektu umožnilo vytvořit flexibilní a rychle upravitelný design bez nutnosti psát samotné CSS.

\section{Shadcn}
Shadcn je sada komponent pro výhradně pro React a Tailwind, která zjednodušuje integraci uživatelského rozhraní do aplikací. Umožňuje rychlé nasazení a styling komponent pomocí Tailwind CSS, čímž snižuje čas potřebný na vývoj frontendu. Jedná se o předem nastylované Radix komponenty.

Narozdíl od ostatních sad komponent jako například Bootstrap, shadcn naimportuje kód komponent přímo do projektu, tudíž jsou plně upravitelné.

Vue komunita vytvořila svou neoficiální verzi Shadcn s názvem Shadcn-vue. 

\section{Clerk}
Clerk je autentizační a správcovská služba uživatelů, která zjednodušuje implementaci přihlašování, registrací a správy uživatelů v aplikacích. Nabízí snadnou integraci s frontendovými technologiemi, jako je Vue.js

V tomto projektu je Clerk použit pro autentikaci, což zajišťuje bezpečný přístup k aplikaci.

\section{Zod}
Zod je knihovna pro validaci dat. Umožňuje definovat schémata, která popisují tvar dat, a následně je validovat.

V tomto projektu je Zod využíván v validace dat přicházející ze shadcn formulářů. Tímto zajišťuje konzistenci a správnost dat na serverové straně.

\section{Docker}
Docker je nástroj, který slouží k vývoji, distribuci a spouštění aplikací. Umožňuje oddělit aplikace od infrastruktury, což usnadňuje a zrychluje nasazování softwaru. Díky Dockeru se dá infrastruktura spravovat podobně jako aplikace. Pomocí metod, které Docker nabízí pro testování, nasazování a distribuci kódu, se dá zkrátit čas mezi napsáním kódu a jeho spuštěním v ostrém provozu\cite{a2024_what}.

V rámci tohoto projektu Docker poskytuje efektivní a rychlé řešení, které umožňuje vývoj a distribuci aplikace napříč různými platformami.

\section{Git / Github}
Git je distribuovaný systém pro správu verzí, který umožňuje efektivní sledování změn a správu různých verzí souborů. Je široce využíván díky své flexibilitě a podpoře týmové spolupráce, čímž se stal standardem mezi verzovacími systémy \cite{enwiki:1261616505}.

V na tomto projektu byl Git nasazen od počátku jako klíčový nástroj pro verzování. Pro sdílení a správu repozitáře byl zvolen GitHub, populární platforma rozšiřující Git o funkce pro ukládání, správu a spolupráci v cloudu. Díky svému intuitivnímu prostředí se GitHub stal oblíbeným řešením pro projekty všech velikostí \cite{enwiki:1262043674}.

\section{CICD - Github actions}
GitHub Actions je platforma pro kontinuální integraci a doručování (CI/CD), která umožňuje automatizovat procesy, jako jsou sestavení, testování a nasazování aplikací. S GitHub Actions lze vytvářet pracovní postupy, které automaticky sestaví a otestují každou pull requestu v repozitáři, nebo nasadí sloučené pull requesty přímo do produkce\cite{githubinc_2024_understanding}.

Při vývoji rezervačního systému byly github actions především použity pro vytvoření docker kontejneru a pro spuštění testů

\section{Python a pytest}
Python je vysoce úrovňový interpretovaný skriptovací jazyk, který se vyznačuje unikátním způsobem strukturování kódu. Namísto tradičního používání složených závorek pro seskupování bloků kódu využívá Python bílých znaků (whitespace), což přispívá k jeho čitelnosti a jednoduchosti \cite{enwiki:1262860862, enwiki:1259453401}.

V rámci tohoto projektu je Python využit pro vývoj API testů, přičemž byl využit framework pytest. Pytest je vcelku jednoduchý a s kombinací s možností rychlého vývoje v Pythonu nám toto teechnologické rozhodnutí umožnilo opravdu rychlý vývoj jednoduchých testů.


\chapter{Teoretická analýza}
Tato kapitola se věnuje teoretickému návrhu systému, jehož jádrem je optimalizace rezervací tréninkových časů a přístrojů v posilovně. Hlavním cílem je definovat algoritmické principy pro tři klíčové funkce.

\begin{enumerate}
    \item \textbf{Doporučení přístrojů} - Systém musí umět doporučovat posilovací přístroje na základě ostatních již zvolených zařízení.
    \item \textbf{Generování času pro trénink} - Systém musí dokázat vytvořit časové alokace pro jednotlivé přístroje vycházející ze zadaného počátečního času, s ohledem na jejich kapacitní limity
    \item \textbf{Doporučení nejbližších časů} - Při manuální konfiguraci časových oken u jednotlivých přístrojů je nutné průběžně detekovat kolize s ostatními rezervacemi a na základě této analýzy doporučit nejbližší volné časové okno stejné délky
\end{enumerate}

Před vysvětnením je potřeba ustanovit si jistou terminologii, která nám popíše určité věci.
\begin{enumerate}
    \item Rezervace - Jedná se o konkrétní rezervaci, je vázaná na datum, plán a uživatele, může mít asociaci s trenérem
    \item Plán - Plán reprezentuje tréninkový plán, který zákazník bude provádět, jsou zde zahrnuty přístroje, které bude využívat, časová okna s nimi spojené, typ tréninku a pod.
    \item Přístroj - Konkrétní přístroj na posilování
\end{enumerate}

Nyní je potřeba si definovat postup, jakým bude uživatel interagovat s těmito algoritmy. Vytvoření rezervace bude probíhat skrze konfigurační formulář, který bude wrapovat tento následující postup, viz Activity diagram procesu vytvoření rezervace. \ref{fig:ReservationActionDiagram}
\begin{itemize}
  \item Zákazník vyplní základní informace ohledně rezervace.
    \begin{itemize}
      \item Mezi základní informace se započítává například datum rezervace, název plánu a počet osob, atd.
    \end{itemize}
  \item Zákazník může vybrat trenéra
  \item Systém vyhledá všechny přístroje dostupné v posilovně
  \item Zákazník vybírá přístroje dokud nevybere všechny které chce.
    \begin{itemize}
        \item Pokud zákazník vybere více než nějaký určený počet přístrojů, tak systém začne doporučovat přístroje, jež jsou spojeny s typem tréninku, který v dosavadním výběru přístroju převažuje. V tomto kroku se odehrává \textbf{Doporučení přístrojů}
    \end{itemize}
  \item Zákazník si zvolí zda chce konfigurovat časová okna manuálně
      \begin{itemize}
        \item Pokud ano, tak systém zablokuje možnost vybírat vybrat pořadí. Zákazník si vybere počáteční čas a systém vygeneruje jednotlivá časová okna, které následně doplní do jednotlivých položek.Zde se odehrává \textbf{Generování času pro trénink}. Dále může časová okna ještě poupravit. 
        \item Pokud ne, tak uživatel zvolí pořadí přístrojů. Následně zákazník volí časová okna pro jednotlivé přístroje. 
    \end{itemize}
    
  \item Během manuálního vkládání/upravování časových oken systém kontroluje, zda v zadaném časovém okně je kolize s jinou rezervací. Pokud je, tak systém vyhledá nejbližší volné stejné délky před a po zvoleném. Zde se bude odehrávat \textbf{Doporučení najbližšího dostupného času}.
  \item Nakonec zákazník zvolí dodatečné informace, jako například kategorie tréninku atd.
\end{itemize}

\begin{figure}
    \centering
    \includegraphics[width=.5\textwidth]{Figures/Bakalarka, rezervace.jpg}
    \caption{Action diagram procesu vytvoření rezervace}
    \label{fig:ReservationActionDiagram}
\end{figure}

\section{Doporučení přístrojů}
Doporučení přístrojů na základě jejich typu je vcelku jednoduché. Nejprve je nutné určit pro jaký typ tréninku se budou přístroje doporučovat. To zajistíme tak, že budeme mít v informaci o možných typech tréninku někde uloženou, konkrétně v databázi. 

Systém tedy projde každý vybraný přístroj, zjistí jaké typy tréninku jsou na přístroji možné a ten, který se vyskytuje ve všech bude považovat za cílový trénink, který zákazník chce navolit. Na základě naleznutého typu tréninku tedy systém najde všechny přístroje, který tento tyt sdílí a podle jejich populaliry vybere několik nejpopulárnějších. 

Problémy s kolizemi v tuto chvíli není potřeba u doporučování přístrojů řešit. Systém nebude mít ve chvíli kdy bude zákazník přístroje vybírat kontext o tom, kdy se trénink bude odehrávat.

\section{Doporučení najbližšího dostupného času}
Doporučení nejbližšího času je již komplexnější. Problematika spočívá v detekci kolizí a v hledání časového okna, které je stejné délky a nekoliduje s žádnou další rezervací a to jak před zvoleným časovým oknem, tak i po něm.

Uživatel bude schopen si také povolit, nebo zakázat kolize na jednotlivých přístrojích. Například je pro systém žádoucí mít neustále povolené kolize na činkách, jelikož jich je mnoho. Kolize se budou řešit úplně stejně jako případy bez kolizí, jen s tím rozdílem, že u kolizí ještě bude systém kontrolovat kapacitu. Pokud bude souběžný počet lidí na jednom přístroji v mezích jeho kapacity, tak systém považuje toto časové okno jako volné; pokud ne, tak ho považuje jako zabrané. Uživatel bude na existující kolizi upozorněn, ale nebude se jednat o stav blokující ho ve vytvoření rezervace.

Systém nejprve načte existující rezervace daného přístroje pro zvolené datum a extrahuje z nich počáteční a koncové časy všech rezervovaných časových oken. Tato okna jsou následně seřazena vzestupně podle počátečního času, čímž vzniká chronologický přehled obsazených intervalů. Před zahájením analýzy systém předpokládá, že uživatelem zadané časové okno splňuje základní validaci – počáteční čas je chronologicky před koncovým časem.

Následně systém porovná vstupní časové okno s uspořádaným seznamem rezervací a identifikuje první interval, který s ním koliduje (překrývá se). Jakmile je kolize detekována, algoritmus vyhledá nejbližší volný interval mezi již existujícími rezervacemi před konfliktním oknem, jehož délka odpovídá nebo přesahuje požadovanou dobu trvání. Stejný postup aplikuje i na intervaly za konfliktním oknem, čímž získá dvě potenciální alternativy: první volné okno před a druhé po původně zvoleném čase.

Pro lepší představu se lze odkázat na diagram \ref{fig:ReservationTimeSuggestionDiagram}. Každá bílá buňka reprezentuje zabraná časová okna, červená ty, se kterými zadaný interval koliduje, šedá ty, která jsou moc krátká, a zelená ty, které jsou vhodné pro doporučení. 

\begin{figure}
    \centering
    \includegraphics[width=1\textwidth]{Figures/time_suggestion_diagram.jpg}
    \caption{Diagram nalezení nejbližších volných časových oken}
    \label{fig:ReservationTimeSuggestionDiagram}
\end{figure}

\section{Generování času pro trénink}
Generování tréninků je z předem zmíněných možností asi nejkomplexnější. Problematika spočívá v tom, že nejenže musíme najít vhodné časy, ale opět řešit kolize a to vše, aby byl trénink bez prázdných časových oken, ve kterých zákazník nemá zarezervovaný nějaký přístroj. 

Nejprve je potřeba systém v určitých směrech omezit. Například bude velmi složité a neefektivní nechat uživatele zvolit pořadí přístrojů, v tu chvíli by ne jen nabylo řešení na komplexitě, ale také by to omezilo systém v ohledu flexibility, což má být jeden z jeho klíčových aspektů. Zároveň nám zablokování možnosti měnit pořadí otevírá dveře k více flexibilnímu řešení, u kterého může pořadí přístrojů měnit sám systém tak, aby výsledný trénink byl časově efektivní a zabránil tvorbě mezer.

Dále systém umožní uživateli si vypnout, či zapnout kolize s ostatními lidmi, co kolize zapnuté mají. Tyto kolize jsou řešeny podobně jako u Doporučení nejbližšího dostupného času. Nakonec dostaneme od uživatele počáteční čas. 

Generování časových oken bude řešeno pomocí grafů s neváženými směrovými hranami, kde každý vrchol grafu nám reprezentuje časové okno. Výsledkem je několik průchodů tohoto grafu (možností pro uživatele). Graf budeme skládat podle zadaných parametrů, průchod grafem je neměnný. 
\subsection{Tvorba Grafu}

\subsection{Tvorba vrcholů}
Data v grafu vycházejí z využití vybraných přístrojů pro daný den počínaje uživatelem zadaným časem. Je nutné v datech zachovat i ty záznamy, které mají kolize vypnuté, jelikož budou důležité pro tvorbu vrcholů.

Jakmile systém získá záznamy o vytížení přístrojů, je potřeba tato data upravit do požadovaného formátu. Tento formát musí obsahovat počáteční a koncový čas časového okna, který reprezentuje. Dále je potřeba  zachovat informace o přístroji a rezervaci. Dále tato data systém upraví ještě jednou a to tím způsobem, že si doplní záznamy o mezerách, tj. časová okna, přes která nikdo nemá vytvořenou rezervaci pro daný přístroj. Tyto záznamy tedy nebudou obsahovat rezervaci. 

Tyto data budeme tedy následovně upravovat na základě vstupních hodnot poskytnutých uživatelem. Konkrétně, za uživatel povolí kolize s ostatními uživateli, kteří tuto možnost také povolili.

Pokud uživatel kolize nepovolí, tak se nabízí zcela jednoduché řešení. Z předem připravených dat jednoduše odstraníme ty možnosti, které obsahují rezervaci, tj. někdo přes toto časové okno již rezervaci má. Tyto data budou tedy výslednými vrcholy pro tento scénář.

V případě, že uživatel kolize povolí, bude problém o něco složitější. Nejprve z právě těch záznamů, které obsahují rezervaci a zároveň mají kolize povolené, systém musí danou rezervaci odstranit, tímto se z těchto záznamů stanou předem zmíněné mezery. Rezervace, které mají kolize zakázané, musí v datech zůstat. Dále systém projde všechny tyto mezery a spojí je dohromady viz obrázek. % TODO
Tímto se nám zmenší počet vrcholů, přes které systém poté bude procházet. Až po spojení mezer systém odstraní záznamy, jež obsahují zbývající rezervace.

Výsledkem obou případů jsou volná časová okna, která jsou pro danou konfiguraci dostupná, právě tyto hodnoty budou hodnotou vrcholů, jež v grafu budou.

\subsection{Tvorba vrcholů}
Pro tvorbu vytvoření hran v grafu se řídí dvěma základními podmínkami.

Následující podmínky využívají 2 vrcholy, vrchol A je právě ten vrchol, ze kterého má směřovat hrana a vrchol B je kandidát na vrchol do kterého hrana směřuje.
\begin{enumerate}
    \item Koncový čas vrcholu B musí být větší, než počáteční čas vrcholu A.
    \item Přístroj asociovaný s vrcholem A nesmí být stejný jako přístroj asociovaný s vrcholem B
\end{enumerate}

Pokud tyto dvě podmínky systém aplikuje na všechna nalezená časová okna, tak výsledkem budou potřebné hrany, skrz které se bude algoritmus pro průchod pohybovat.


\subsection{Průchod grafem}
Na průchod grafem se nabízí spousta možností, jako jsou například: DFS, BFS, Dijkstra a pod. Tyto algoritmy se dají využít na spoustu use-case, ale pro naši implementaci si vystačíme s vlastním řešením.

Průchod grafem bude řešen pomocí rekurze. Princip rozhodování dalšího vrcholu bude jednoduchý. Následující vrchol bude první prvek z množiny vzestupně seřazené podle koncového času, který nebyl dosud navštíven a zároveň neobsahuje přístroj, jenž byl navštíven a časově se do vrcholu vejde.


\begin{figure}
	\centering
	\subfloat[neorientovaný graf\label{fig:Subfig1}]
	{
		\includegraphics[width=0.45\textwidth]{Figures/graph-node-1.png}
	}
	\hspace{3em} % make more space
	\subfloat[reprezentace grafu\label{fig:Subfig2}]
	{
		\includegraphics[width=0.45\textwidth]{Figures/graph-node-2.png}
	}
	\caption{Ukázkový obrázek se dvěma podobrázky}
	\label{fig:TopLevelFigureLabel}
\end{figure}

\chapter{Infrastruktura}

\section{Úvod}
S výjimkou databázového systému je aplikace rozdělena do dvou oddělených vrstev: frontend a backend. Rozhodnutí nevyužít fullstack framework, jako je například Nuxt, je poměrně jednoduché. Tyto dvě vrstvy mohou fungovat zcela odděleně, což umožňuje využití různých verzí technologií a zajišťuje větší flexibilitu při správě a nasazování jednotlivých částí systému.

Toto rozdělení přináší také zásadní výhody z hlediska budoucí rozšiřitelnosti. Aplikace se dá snadno přizpůsobit pro různé platformy – například ji lze rozšířit o mobilní nebo desktopovou aplikaci, aniž by to narušilo stávající strukturu. Díky této modularitě je architektura systému robustnější a lépe připravená na budoucí požadavky. Navzdory rozdělení je projekt uložený v jednom git repozitáři.

\section{Databáze}

% \subsection{Specifikace zadání}

\subsection{Vize}
Navrhovaný systém pro posilovnu bude zajišťovat správu rezervací, uživatelů a vybavení posilovny. Hlavním účelem systému je rezervační konfigurátor, jehož primární funkcí je alokace dostupných přístrojů. Systém integruje algoritmus pro dynamické doporučení optimálních časových intervalů na základě vytíženosti zařízení a přizpůsobený výběr přístrojů na bázi typu tréninku.

\subsection{Role}
Rolí systém podporuje několik. Uživatele rozdělujeme do tří skupin.

\begin{description}
    \item[Zákazník] je role s nejnižšími právy. Zákazník bude representovat osobu, která si v systému bude chtít vytvořit rezervaci
    \item[Trenér] je role která má rozšítěný přístup k systému, je to obyčejný uživatel, který může být přiřazen k rezervaci. Asociaci s rezervací může ale sám zrušit. V zadání tuto roli nemáme popsanou, tudíž se pro tento typ účtu nebude v rámci této bakalářské práce implementovat logika navíc. Prozatím bude fungovat jako obyčejný uživatel, kterého lze přidat k rezervaci jako trenéra.
    \item[Zaměstnanec] je role, která má přístup k administraci, je to role s nejvyššími právy a je schopna přidávat, manipulovat a mazat veškerá data. Bude také mít přístup ke všem statistikám a výpisům.
\end{description}
Tyto skupiny nám zajistí integritu přístupu dat a jejich manipulaci. Pro tento systém není potřeba více rolí, jelikož Zaměstnanců, kteří jsou pověřeni manipulací dat a přístupu do administrace je malé množství, tudíž téměř neomezený přístup k datům má nízké chybové dopady.

\subsection{Popis}
Systém bude primárně zaměřen na chod posilovny, což zahrnuje správu rezervací přístrojů na konkrétní časy a potenciálně i sledování výdělků. Pro zajištění těchto specifikací je potřeba se ujistit, že systém tyto specifikace pokrývá. Systém je postaven na třech pilířích.

\begin{description}
    \item[Uživatelské účty] Většina záznamů v systému vyžaduje uchovávání informací o uživateli. Samotná autentizace a související procesy, jako je ověřování uživatelů, budou řešeny mimo rámec databáze pomocí externích služeb. Nicméně je stále potřeba ukládat dodatečné informace o uživatelích přímo v databázi. Pro tento účel systém obsahuje jednoduchou tabulku s názvem account, která slouží k uchování základních informací o uživatelských účtech, jako jsou identifikátory nebo další nezbytné údaje.
    \item[Rezervace a plán] Jak napovídá samotný název, tabulka rezervace reprezentuje uživatelovu rezervaci, která je vždy vázána na konkrétní den. Každá rezervace je spojená s uživatelem a určitým plánem a může mít přiřazeného trenéra. Plán představuje komplexnější strukturu několika tabulek, které uchovávají podrobné informace o rezervaci, například konkrétní přístroje, časové rozvrhy a další detaily. Plán je tvořen následujícími tabulkami: plan, plan\_category a plan\_machine.
    \item[Přístroje a kategorie] Tabulka machine reprezentuje skutečné posilovací přístroje dostupné v posilovně. Každý přístroj má definované základní informace, jako jsou jeho název a popis, a je spojen s určitým plánem. Vztah mezi přístroji a plánem je typu N:M (mnoho na mnoho), což umožňuje asociaci jednoho přístroje s více plány a naopak. Tento vztah je dále rozšířen o dodatečné informace, jako je počet opakování (repetic), počet sérií (setů) a časové intervaly, zahrnující počáteční a koncový čas cvičení. Dále jsou s každým plánem a přístrojem asociovány kategorie, které jsou uchovávány v tabulce plan\_category. Tyto kategorie umožňují lépe strukturovat a organizovat přístroje a plány podle typu cvičení nebo jiných kritérií, což usnadňuje správu systému, zvyšuje jeho přehlednost a poskytne možnost doporučit Přístroje podle předem vybraných.
\end{description}


\subsection{Datová analýza}

\begin{figure}[h!]
	\includegraphics[width=1\textwidth]{Figures/ermodel.png}
	\caption{ Relační model}
	\label{fig:RelationalModel}
\end{figure}

\subsubsection{Popis dat}
Tato sekce se věnuje popisu nejdůležitějších tabulek, které můžeme vidět v ER-modelu. Jedná se o Rezervaci, Přístroje, Plán a "plan\_machine", která reprezentuje jednotlivý přístroj a detaily uživatelské interakce s ním v daném plánu. Ostatní tabulky slouží primárně k uchovávání doplňujících informací nebo ke správě relací mezi výše uvedenými tabulkami.

\begin{table}[h!]
	\caption{Reservation}
    \label{tab:dat-dictionary-reservation}
	\begin{tabular}{|p{3.5cm}|p{2cm}|p{1cm}|p{2.5cm}|p{.75cm}|p{.5cm}|p{3.25cm}|}
		\hline
        \textbf{Název atributu} & \textbf{Dat. Typ} & \textbf{Délka} & \textbf{Klíč} & \textbf{Null} & \textbf{IO} & \textbf{Popis} \\
        \hline
        reservation\_id & INT & - & Primary & Ne & & Id rezervace \\
        \hline
        ammout\_of\_people & INT & - & - & Ne & & Počet lidí \\
        \hline
        reservation\_time & DATE & - & - & Ne & & Čas příchodu \\
        \hline
        customer\_id & INT & - & Cizí (account) & Ne & & Id uživatele, na kterého je rezervace napsaná \\
        \hline
        trainer\_id & INT & - & Cizí (account) & Ne & & Id trenéra \\
        \hline
        plan\_id & INT & - & Cizí (plan) & Ne & & Id plánu \\
        \hline
	\end{tabular}
\end{table}

\begin{table}[h!]
	\caption{Machine}
    \label{tab:dat-dictionary-machine}
	\begin{tabular}{|p{3.5cm}|p{2cm}|p{1cm}|p{2.5cm}|p{.75cm}|p{.5cm}|p{3.25cm}|}
		\hline
        \textbf{Název atributu} & \textbf{Dat. Typ} & \textbf{Délka} & \textbf{Klíč} & \textbf{Null} & \textbf{IO} & \textbf{Popis} \\
        \hline
            machine\_id & INTEGER   &  -    & Primary       & Ne &  & Id stroje \\
            \hline
            machine\_name     & VARCHAR   &  64   & -                 & Ne &  & Název stroje \\
            \hline
            max\_weight       & FLOAT   &  -   & -                 & Ne &  & Maximální váha, kterou lze použít \\
            \hline
            min\_weight       & FLOAT   &  1    & -                 & Ne &  & Minimální váha, kterou lze použít \\
            \hline
            max\_people       & INTEGER   &  -  & -                 & Ne &  & Maximální počet lidí na stroji \\
            \hline
            avg\_time\_taken    & INTEGER   &     & -                 & Ne &  & Průměrný trávený čas \\
            \hline
            popularity\_score & INTEGER      &  -    & -                 & Ne &  &  Počet rezervací \\
        \hline
	\end{tabular}
\end{table}

\begin{table}[h!]
	\caption{plan}
    \label{tab:dat-dictionary-plan}
	\begin{tabular}{|p{3.5cm}|p{2cm}|p{1cm}|p{2.5cm}|p{.75cm}|p{.5cm}|p{3.25cm}|}
		\hline
        \textbf{Název atributu} & \textbf{Dat. Typ} & \textbf{Délka} & \textbf{Klíč} & \textbf{Null} & \textbf{IO} & \textbf{Popis} \\
        \hline
            plan\_id & INTEGER   &  -    & Primary       & Ne &  & Id plánu \\
        \hline
            plan\_name     & VARCHAR   &  64   & -                 & Ne &  & Název plánu \\
        \hline
            account\_id     & INTEGER   &  -   & -                 & Ne &  & Id uživatele, který vytvořil plán \\
        \hline
	\end{tabular}
\end{table}

\begin{table}[h!]
	\caption{plan\_machine}
    \label{tab:dat-dictionary-plan-machine}
	\begin{tabular}{|p{3.5cm}|p{2cm}|p{1cm}|p{2.5cm}|p{.75cm}|p{.5cm}|p{3.25cm}|}
		\hline
        \textbf{Název atributu} & \textbf{Dat. Typ} & \textbf{Délka} & \textbf{Klíč} & \textbf{Null} & \textbf{IO} & \textbf{Popis} \\
        \hline
            plan\_id        & INTEGER   &  -    & Cizí (plan)       & Ne &  & Id plánu \\
        \hline
            machine\_id     & INTEGER   &  -    & Cizí (machine)       & Ne &  & Id stroje \\
        \hline
            sets                & INTEGER   &  -   & -                 & Ne &  & TODO \\
        \hline
            reps                & INTEGER   &  -    & -                 & Ne &  &  TODO \\
            start\_time     & TIME      &  -    & -                 & Ne &  & Čas začátku \\
        \hline
            end\_time       & TIME      &  -    & -                 & Ne &  & čas konce \\
        \hline
            can\_disturb          & BOOLEAN   &  1    & -                 & Ne &  & TODO \\
        \hline
	\end{tabular}
\end{table}

\subsubsection{Procedury a funkce}
Konkrétní procedury a funkce budou popsány v pozdější části této práce. Klíčovým aspektem návrhu je určení, které operace bude efektivnější realizovat na úrovni databáze a které na aplikační vrstvě. Toto rozhodnutí je zásadní, protože ovlivňuje nejen efektivitu a přehlednost řešení, ale také jeho dlouhodobou udržovatelnost.

Procedury a funkce jsou v kontextu navržené architektury využity pro extrakci komplexních datových struktur, nejsou však aplikovýny pro jiné operace jako například manipulace s daty, nebo složitější transakce. Toto rozhodnutí vychází ze dvou východisek:

\begin{description}
    \item[Minimalizace funkcionality databázové vrstvy] 
    Databáze je využita především jako stabilní uložiště dat a optimalizovaný nástroj pro jejich vyhledávání. Transformace dat a obchodní logika je delegována na aplikační vrstvu.
    \item[Využití výhod aplikační vrstvy]
    Výhodou aplikační vrstvy oproti vrstvě datové je možnost využití obecných programovacích jazyků (např. JavaScript), které – na rozdíl od databázového jazyka PL/pgSQL – poskytují širší spektrum nástrojů pro manipulaci s daty, implementaci procedurální logiky a správu komplexních operací.  
\end{description}
Tento přístup přesouvá složitou obchodní logiku z databázové vrstvy do prostředí, které lépe odpovídá požadavkům na moderní softwarový vývoj. Zároveň toto rozhodnutí reflektuje princip separace zodpovědnosti (Separation of Concern - SoC), čímž ustanovuje jasné hranice mezi aplikační a datovou vrstvou, které následně vedou k více robustnímu a bezpečnému řešení. \cite{de2002importance}




\section{Backend}
Jedná se o HTTP rozhraní, které zajišťuje komunikaci mezi databází a uživatelským rozhraním. Rozhraní je napsáno v jazyce TypeScript, což přispívá k jazykové konzistenci napříč celým systémem a umožňuje využití moderních jazykových funkcí, včetně silné typové kontroly.

Pro práci s databází nebyly použity žádné existující knihovny pro implementaci ORM (Object-Relational Mapping). Místo toho jsem se rozhodl vytvořit vlastní implementaci ORM. Tento přístup poskytuje větší flexibilitu a kontrolu nad způsobem, jakým jsou data spravována a manipulována. Konkrétní detaily této implementace budou popsány v následujících sekcích.

\subsection{ORM}
Pro implementaci ORM jsem se rozhodl využít kombinaci reflexe a dekorátorů. Toto řešení mi umožnilo výrazně zjednodušit vývoj jednoduchých CRUD(tj. Create, Read, Update, Delete) operací pro jednotlivé koncové body. Výhodou tohoto přístupu je, že stačí definovat model, opatřit ho vhodnými dekorátory, a samotné mapování na SQL se následně provádí automaticky díky reflexi.

Důvodem rozhodnutí implementace vlastního ORM byla snaha o minimalizaci externích knihoven a jiných závislostí na straně backendu. Tento přístup přináší výhody, jako lepší kontrolu nad kódem, vyšší kvalitu a jednoznačnou odpovědnost za případné chyby.

Konkrétní implementační detaily budou popsány v následující části.

\subsubsection{Dekorátory}
Dekorátory umožňují rozšířit funkcionalitu běžných tříd pomocí metadat nebo různých nadstaveb, které „dekorují“ danou třídu nebo její atributy\cite{TSDecorators}. V kontextu této práce byly tyto dekorátory implementovány přímo pro potřeby projektu, zejména pro přesné mapování objektových modelů na databázové tabulky a sloupce. Mezi implementované dekorátory vytvořené přímo pro tuto architekturu lze rozlišit několik typů:
\begin{description}
    \item[Table] 
    Tento dekorátor slouží k přiřazení názvu tabulky dané třídě. Pomocí reflexe je název tabulky propojen s třídou, která ji reprezentuje, což usnadňuje práci při mapování modelů na SQL.
    \item[PrimaryKey]  
Dekorátor, který specifikuje primární klíč tabulky. Aplikace jej využívá pro identifikaci a práci s primárním klíčem záznamů v databázi.

\item[Column] 
Tento dekorátor přidává k atributům třídy odpovídající názvy sloupců v databázi. Zároveň přiřazuje daným atributům i specifické datové struktury v pozadí.

\item[ForeignKey] 
Dekorátor určený pro označení cizího klíče. Do dekorátoru se předává typ reprezentující odkazovanou tabulku, přičemž atribut, na který je dekorátor aplikován, musí být stejného typu. Navíc je nutné, aby odkazovaný typ dědil z obecné třídy \texttt{Model}.

\item[UnInsertable]
Označuje atributy, které mají být při vkládání dat do databáze ignorovány. Je využíváno pro zamezení vložení atributu, jejichž hodnota je nastavována skrze základní hodnoty v databázi. Například create\_date v tabulce account.

\item[DifferentlyNamedForeignKey]  
Používá se v případě, kdy atribut reprezentuje FK ale jeho název se liší od názvu primárního klíče v odkazované tabulce. Tento dekorátor řeší situace, kdy databázová struktura neodpovídá standardnímu pojmenování. Například mějme FK \texttt{customer\_id}, který odkazuje na tabulku \texttt{account}, jenže PK má v dané tabulce název \texttt{account\_id}.

\item[UnUpdatable]  
Podobně jako \texttt{UnInsertable} tento dekorátor zajišťuje, že atributy označené tímto dekorátorem budou ignorovány při požadavku aktualizace dat.

\item[ManyToMany]
Dekorátor, který označuje atributy reprezentující vztah typu N:M (mnoho na mnoho) mezi dvěma tabulkami. Tento dekorátor umožňuje automatické zpracování spojovacích tabulek a souvisejících operací.
\end{description}


\subsection{Struktura}
Struktura projektu prošla několika výraznými změnami. S každou iterací systém nabýval na komplexitě a více strádal na přehlednosti, což vedlo k rozhodnutí provést kompletní přepsání backendu. Toto rozhodnutí vedlo k výraznému zlepšení přehledu a jednoduchosti systému. Podrobné poznámky k procesu refactoringu a jednotlivým iteracím jsou uvedeny v následujících částech textu.

Aktuální backendová architektura je koncipována jako přehledně strukturované řešení s explicitním důrazem na logické uspořádání komponent. Její organizace, založená na seskupování zdrojů dle koncových bodů (např. rezervace, uživatelský účet, atd.), místo tradičního rozdělení podle funkcí (např. service, controller, atd.)
Kořenový adresář je uspořádán následujícím způsobem:
\input{Chapters/Infrastructure/Backend/structure/FileTree}
\subsection{Root složka}
\begin{description}
    \item[db] 
    Zde se ukládají SQL scripty pro vytvoření databáze, vkládání testovacích dat další podobné scripty.
    \item[tests]  
    Zde jsou uloženy testy napsány v pythonu pomocí knihovny pytest. Stručný popis testování můžete nalést v kapitole "Testy a CICD"
    \item[src] 
    V této složce se nachází veškerá logika aplikace, která je podrobně popsána níže.
\end{description}

\subsection{Složka src}
Složka src centralizuje veškerou logiku Express.js HTTP serveru a je strukturována do následujících klíčových modulů:
\begin{description}
    \item[database] 
    Složka database obsahuje logiku specifickou pro ORM a obecné databázové operace. Jsou zde definovány nástroje, jako například výše zmíněné dekorátory pro mapování dat. Nachází se zde také typy reprezentující odpovědi z databáze, což zjednodušuje práci při manipulaci s daty.
    \item[endpoints]
    Každý koncový bod v projektu obsahuje následující soubory: controller, error-handler, model, routes a service. V některých případech může obsahovat i část database, která je zodpovědná za specifické databázové operace, jako jsou volání procedur nebo složité transakce.
    \item[errors] 
    Složka errors obsahuje základní error handler, ze kterého vycházejí ostatní handlery. Dále obsahuje definici typu CodedError, což je vlastní typ chyby obsahující kód chyby a odpovídající hlášku. Součástí je také ErrorCode, což je výčet chyb, které mohou nastat.
    \item[request-utility] 
    Složka request-utility obsahuje definice těl HTTP odpovědí, které se vrací z controllerů. Patří sem například:
    CreatedResponse pro úspěšné POST požadavky.
    DeletedResponse pro úspěšné DELETE požadavky.
    \item[router] 
    Router definuje všechny HTTP cesty, které API nabízí, a volá specifické routery pro jednotlivé endpointy.
    \item[utils] 
    Složka utils obsahuje pomocné funkce a typy, které usnadňují práci. Je zde například typ IDictionary (typová definice mapy pro libovolné datové typy), funkce safeAwait (funkce pro zjednodušení práce s try-catch. Umožňuje práci s chybami v podobném stylu jako v jazyce Go, což přispívá k přehlednosti kódu), definici loggeru (využívá PINO knihovnu) a pod.
    \item[app.ts] 
    Soubor app.ts slouží jako vstupní bod aplikace. Jeho hlavní funkcí je vytvoření instance frameworku ExpressTs
\end{description} 

\subsection{Endpoints / Koncové body}
V další části popíšeme strukturu modulů a na příkladech ukážeme tok dat od rout přes aplikační logiku k databázi a zpět ke klientovi:

\subsection{model}
Reprezentuje data vrácená v odpovědi. Model využívá dekorátory, které definují mapování mezi daty v modelu a databází. Jedná se o třídu, která dědí z obecného modelu.

\lstinputlisting[label=src:BERoutes,caption={Příklad implementace endpoint modelu}]{SourceCodes/foo.model.ts}

Tento příklad demonstruje definiční datový model Foo s následující strukturou:
\begin{description}
    \item[FooId (typ number)] automaticky generovaný primární klíč tabulky
    \item[Bar (typ Bar)] cizí klíč definující 1:1 asociaci s tabulkou Bar.
\end{description} 

Model zároveň ilustruje aplikaci dekorátorů (např. @PrimaryKey, @ForeignKey, @Column, atd.), které rozšiřují funkcionalitu modelu o metadata nutná pro obecné požadavky na databázi. 

Následující příklad bude popisován dále v popisech dalších bodech.
\subsection{routes}
Obsahuje definice HTTP cest, které se volají v globálním routeru. Tyto definice obvykle pouze směrují požadavky na příslušný controller.

\lstinputlisting[label=src:BERoutes,caption={Příklad implementace endpoint routů}]{SourceCodes/foo.routes.ts}

V tomto příkladu nejprve vytvoříme instanci express.js routeru. Následovně definujeme jednotlivé koncové body, které budou pro uživatele dostupné. Tyto definice pouze volají příslušný controller, kterému předává informace o requestu (dotazu) a o požadované response (odpovědi)

\subsection{controller} 
Zajišťuje logiku volání příslušných služeb (service) a sestavuje tělo HTTP odpovědi na základě výsledků těchto služeb.

\lstinputlisting[label=src:BEControllers,caption={Příklad implementace endpoint controllerů}]{SourceCodes/foo.controller.ts}

Kontroler FooController využívá safeAwait pro asynchronní error handling bez try/catch, inspirovaný Go jazykem. Metoda FindById nejprve načte entitu Foo, poté asociovaný Bar, a standardizuje odpovědi přes OkResponse/FailedResponse.

\subsection{service}
Úloha service je na základě vstupních dat vrátit data namapovaná na požadovaný typ. Tyto data je možné získat z databáze, nebo jiných služeb.

\lstinputlisting[label=src:BEServices,caption={Příklad implementace endpoint service}]{SourceCodes/foo.service.ts}

Příklad FooService ukazuje, jak metoda GetFooById využívá BasicQueryDatabase pro komunikaci s databází a safeAwait ke zpracování chyb bez nutnosti zanořených try/catch bloků. Surová data z dotazu jsou transformována do instance třídy Foo pomocí explicitního mapování, které zajišťuje validaci a typovou bezpečnost.

\subsection{database}
Pokud nejsou obecné operace pro práci v databázi pro potřebnou akci dostatečné, lze vytvořit instance databáze, která se od obecného řešení odvíjí, ale zároveň implementuje specifickou logiku vyžadovanou danými požadavky.

\lstinputlisting[label=src:BEServices,caption={Příklad implementace endpoint service}]{SourceCodes/bd-select.ts}

SelectSpecific je jedna z funkcí třídy BasicQueryDatabase, která využívá metadata navázaná na objekt pomocí dekorátorů. Jak lze vidět funkce se převážne jen stará o sestavení SQL dotazu, vrácení získaných dat a uzavření spojení s databází.

\subsection{error-handler}
Slouží jako mapper, který na základě vrácených chyb sestavuje odpovídající HTTP odpovědi. Pod pokličkou se jedná o hash mapu, která obsahuje typy errorů se status kódy. Je volán z controlleru.

% \subsection{Příklad}

\subsubsection{Model}
\lstinputlisting[label=src:BEModel,caption={Příklad implementace endpoint modelu}]{SourceCodes/foo.model.ts}

V ukázce je možné vidět praktické použití dekorátorů. Data, která potřebujeme inicializovat, jsou naplněna prostřednictvím konstruktoru. Konstruktor přijímá parametr typu IDictionary<any>, což je v podstatě HashMap, který umožňuje uchovávat hodnoty libovolného datového typu.


\subsubsection{Routes}

\subsubsection{Controller}


TODO

\subsubsection{Service}
\lstinputlisting[label=src:BEService,caption={TODO}]{SourceCodes/foo.service.ts}

TODO

\subsubsection{Database}
\lstinputlisting[label=src:BEDB,caption={TODO}]{SourceCodes/bd-select.ts}

TODO

\section{Frontend}
Frontend je uživatelské rozhraní aplikace, které interaguje s uživatelem a zobrazuje data z backendu. Je napsán ve Vue.js, což je moderní JavaScriptový framework pro vytváření uživatelských rozhraní.

\subsubsection{Struktura}

\subsubsection{Příklad}
\chapter{Implementace}
V kontextu již definovaného implementačního rámce (technologický stack, datové toky, architektura) přistupujeme k analýze klíčových algoritmických komponent systému: doporučování časů přístrojů, generování časů pro tréninky způsoby a řešení kolizí.

\section{Doporučení přístrojů}
Jak již bylo popsáno v kapitole \textbf{Teoretická analýza}, tak koncept doporučení přístrojů není příliž složitý. Co se praktické implementace týče, tak je to obdobné.

\subsection{Frontend}
Nejprve je potřeba aby uživatel přístroje vybral. K tomu poslouží řada checkboxů, které tyto přístroje budou reprezentovat. Checkboxy mohou působit jako ne příliš dobré řešení, ale pro demostrační účely této bakalářské práce to bude stačit.

Ve výsledku budou přístroje uloženy v poli. Toto pole bude monitorováno pomocí computed property, která v sobě bude uchovávat informaci ohledně nejčastějším id kategorie nalezeným v těhto přístrojích.

Computed properties (česky: Vypočítané vlastnosti) jsou vlastnosti, které jsou vypočtené z jiných reakticvích proměnných. Jejich největší výhoda spočívá v sebekontrolování jejich závislostí, t.j. jakmile se jakákoli reaktivní proměnná, na které je vypočítáná vlastnost závislá změní, hodnota této vlastnosti je přepočítá. Zde tato vlastnost nabývá podoby proměnné mostFrequentCategoryId. Computed property je definována pomocí definice getter funkce. Tato computed property bude sloužit při následnému volání na API, které se uskuteční pomocí watcheru.

Podobně jako u computed vlastností je watch způsob jakým lze reagovat na změnu reaktivní proměnné, ale s tím rozdílem, že watcher nedrží žádnou hodnotu. Watcher pouze sleduje reaktivní proměnnou a při její změně zavolá callback funkci. Watcher by se dal jednoduše vysvětlit jako event handler pro reaktivní proměnné ve Vue.js.

\begin{lstlisting}
const selectedMachines = ref<Machine[]>([])
const recommendMachines = ref<Machine[]>([])

const mostFrequentCategoryId = computed(() => {
  // Find the and return the categoryId with the most occurences
});

watch(
  mostFrequentCategoryId,
  async (newId) => {
    // Get the recommended machines from the API and assign them to recommendedMachines
  },
  { deep: true }
)
\end{lstlisting}

Kombinací těchto dvou konceptů docílíme menší frekvenci dotazů na API, čož umožnuje menší zatížení backendu a menší potřebu překreslovat doporučené přístroje na Frontendu, což by bylo způsobeno neustálým přepisováním hodnot reaktivních proměnných, jež jsou v template sekci volány.

\subsection{Backend}
Ze strany backendu se jedná o dotaz na databázi. Tento dotaz je poněkud komplikovanější. Má více vstupních parametrů a zároveň jeho implementace požaduje vnořené příkazy. Jak již bylo zmíněno v popisu databáze, tak pro tyto případy bylo učiněno rozhodnutí takovéto příkazy zaobalit do procedury přímo v databázi. V implementaci API se tedy volá pouze tato procedura.

\begin{lstlisting}
const result: Machine[] = await this.sql<Machine[]>`
    SELECT * 
    FROM get_machines_in_same_category(${id})
`
\end{lstlisting}

Volaná procedura tedy obsahuje pouze následující SQL dotaz. Tento dotaz nám propojuje 3 tabulky. Machine (reprezentuje individuální přístroj, např. Bench, Squat rack, atd.), machine exercise type (reprezentuje N:M vztah mezi cvikem a přístrojem) a exercise type (reprezentuje cvik, např. Dřep, Biceps curl, atd.). Výsledek spojených tabulek je následně profiltrován pomocí vnořeného SQL dotazu, který vyhledává první id kategorie, které přístroj s vloženým id má. Následně se z výsledku předešlé filtrace vyjme přístroj s id v parametru procedury. Nazávěr se data sestupně seřadí podle jejich popularity.

\begin{lstlisting}[language=SQL]
SELECT DISTINCT m.*
FROM machine m
JOIN machine_exercise_type met ON m.machine_id = met.machine_id
JOIN exercise_type et ON met.exercise_type_id = et.exercise_type_id
WHERE et.category_id = (
    SELECT et.category_id
    FROM machine_exercise_type met
    JOIN exercise_type et ON met.exercise_type_id = et.exercise_type_id
    WHERE met.machine_id = input_machine_id
    LIMIT 1
)
AND m.machine_id != input_machine_id
ORDER BY m.popularity_score DESC;
\end{lstlisting}
Přirozeně se o tomto řešení dá říci, že je zbytečně komplikované a jelikož hledáme nejčastější kategorii na straně frontendu. Toto sebou ale nese jeden problém. Konkrétně se jedná o ten fakt, že doporučení pomocí přístroje nám nabízí více možností z ohledu dalšího rozšíření systému. Dále může jeden přístroj mít více kategorií, tudíž nemůžeme vědět jakou konkrétně chceme najít. Hledání kategorie sloužilo pouze pro vymezení možností v rámci tréninku a ne v rámci individuálních přístrojů.

\section{Doporučení najbližšího dostupného času}

Pro doporučení nejbližšího dostupného času se musí vzít v potaz kolize s jinými rezervacemi, což je problematika, která byla popsaná v teoretické analýze. Co popsáno ale nebylo je způsob jak tuto kolizi nalezneme v reálném čase, t.j. ve chvíli, kdy definujeme časové okno.

\subsection{Hledání kolizí v reálném čase}
Tuto část rozebíráme v praktické implementaci jelikož je řešení velmi úzce spojené s praktickou implementací na frontendu. 

Jako jednoduché řešení se nabízí po každé změně časového okna poslat dotaz skrze API, který by vrátil počet rezervací s vypnutými kolizemi v daný čas a pokud by byl počet větší, nebo roven jedné, tak bychom časové okno označili jako kolidující. Tento způsob by ale způsobil obrovské množství dotazů na API, což by způsobilo vysoké zatížení backendu. To by se částečně dalo vyřešít využitím WebSockets, nebo podobnými způsoby.

Ačkoli pokusy o funkčnost výše zmíněného řešení jsou lákavé v ohledu na fascinaci s jinými řešeními, nabízí se také mnohem intuitivnější řešení. Při pokročení na krok konfigurace časového okna systém syšle dotaz na backend o aktuální rezervace pro daný den, pro každý vybraný přístroj. Tyto data budou následně používány při validaci jednotlivých časových oken pomocí zod validačních schémat. Pokud systém kolizi při validaci najde, tak zašle dotaz pro vyhledání nejbližších časů pro daný přístroj.


\subsection{Backend}
Na backendu algoritmus pracuje s daty získanými z databáze, která obsahují nejen plány rezervací pro daný přístroj, ale také informace o časových oknech předchozích/následujících rezervací a kapacitních omezeních. Cílem je najít nejbližší volný interval stejné délky jako požadované časové okno, a to buď před první kolidující rezervací, nebo za poslední kolidující rezervací, s ohledem na možnost povolení kolizí.

Data jsou získávána pomocí následujícího sql dotazu:
\begin{lstlisting}[language=SQL]
SELECT pm.*,
  CASE
    WHEN
      ((pm.previous_start_time <= pm.end_time AND pm.previous_end_time >= pm.start_time) OR
      (pm.previous_start_time <= pm.next_end_time AND pm.previous_end_time >= pm.next_start_time) OR
        (pm.start_time <= pm.next_end_time AND pm.end_time >= pm.next_start_time))
      AND m.max_people >= SUM(r.amount_of_people) + input_amount_of_people
  THEN true
  ELSE false
END AS can_fit
FROM get_plan_machines_with_next_and_prev(input_machine_id, input_reservation_date) pm
  INNER JOIN reservation r ON r.plan_id = pm.plan_id
  INNER JOIN machine m ON m.machine_id = pm.machine_id
GROUP BY
  pm.plan_id, pm.machine_id, pm.can_disturb, pm.start_time, pm.end_time,
  pm.previous_plan_id, pm.previous_start_time, pm.previous_end_time,
  pm.next_plan_id, pm.next_start_time, pm.next_end_time,
  r.amount_of_people, m.max_people;
END;
\end{lstlisting}
Jak si lze všimnout tento SQL dotaz vyhledává výsledky jiné procedury. Vráceným vásledkem této procedury je každá rezervace společně s časovými okny předchozí a následující rezervace.
Dotaz také zjistí, zda se počet lidí ve vytvářené rezervaci vleze do kolidovaných rezervací s ohledem na kapacitu daného přístroje. Výsledek těhto podmínek je booleanová hodnota can\_fit

Dalším krokem po získání dat z databáze je aplikovat algoritmus implementovaný v metodě \texttt{SuggestTimes}, která se nachází ve třídě \texttt{MachineService}. Tento algoritmus následuje tyto kroky:

\begin{lstlisting}
static async SuggestTimes(
    machineUsage: MachineUsage[],
    desiredTimeRange: TimeRange,
    canDisturb: boolean
): Promise<Suggestion> {
    ...
}
\end{lstlisting}

Tato funkce příjmá výsledek dotazu na databázi v podobě pole v proměnné \texttt{machineUsage}, \texttt{desiredTimeRange} reprezentující žádáné časové okno a \texttt{canDisturb}, jež značí povolení kolizí. Metoda vrací typ \texttt{Suggestion}.

\texttt{Sugestion} obsahuje 2 časová okna, jedno před kolidujícím a druhé za ním. Typ \texttt{TimeRange} reprentuje časové okno samotné a tvoří jej 2 časové záznamy (Start, End) a booleanová hodnota značící existenci kolize.

\begin{lstlisting}
interface TimeRange {
    StartTime: Time
    EndTime: Time
    isColiding: boolean
}

interface Suggestion {
    PrevSuggestion: TimeRange
    NextSuggestion: TimeRange
}
\end{lstlisting}
\begin{enumerate}
    \item \textbf{Detekce kolizí}
	Požadované časové okno a všechny rezervace se převedou na minutovou reprezentaci (např. 14:30 → 870 minut). Kolize se detekuje pomocí intervalové logiky:
\[
K \exists \iff \max(t_{1start}, t_{2start}) \leq \min(t_{1end}, t_{2end})
\]
	Kdy \( t_{1} \) je čas pro danou rezervaci a \( t_{2} \) je čas pro žádané časové okno.
	Výsledkem je seznam \texttt{colidingPlan} s kolidujícími rezervacemi.
	
	V kódu tato část vypadá nějak takto:
	
    \begin{lstlisting}
const colidingPlan = machineUsage
	.map((x: MachineUsage) => {
	    const result = mapRes(x)
	    return findPlanFull(result)
	})
	.find((time) => {
	    const {StartTime, EndTime}  = time
	    const endTimeInMinutes = EndTime.hour * 60 + EndTime.minute
	    const startTimeInMinutes = StartTime.hour * 60 + StartTime.minute
	    const minEndTime =
		    endTimeInMinutes <= desiredEndInt ? endTimeInMinutes : desiredEndInt
	    const maxStartTime =
		    startTimeInMinutes >= desiredStartInt ? startInt : startTimeInMinutes
	    return maxStartTime <= minEndTime
	})
    \end{lstlisting}

    \item \textbf{Výpočet potenciálních časových oken}
    \begin{itemize}
    	\item Před kolidující rezervaci: Pokud mezi koncem předchozí rezervace a začátkem aktuální rezervace existuje mezera větší nebo rovna délce požadovaného okna, vypočítá se okno jako \texttt{startTimeInMinutes - duration}.
    	\item Po kolidující rezervaci: Pokud mezi koncem aktuální rezervace a začátkem následující rezervace existuje dostatečná mezera, vypočítá se okno jako \texttt{endTimeInMinutes}. 
	

    \begin{lstlisting}
// Previous suggestion
if (prevEndTimeInMinutes === null || prevEndTimeInMinutes + duration <= startTimeInMinutes) {
    const startSuggest = startTimeInMinutes - duration
    prevSuggestion = createSuggestion(startSuggest, false)
    prev = prevSuggestion
}

// Next suggestion
if (nextStartTimeInMinutes === null || nextStartTimeInMinutes - duration >= endTimeInMinutes) {
    const startSuggest = endTimeInMinutes
    nextSuggestion = createSuggestion(startSuggest, false)
    next = nextSuggestion
}
    \end{lstlisting}
    \end{itemize}

    \item \textbf{Rekurzivní průchod}
    Pokud žádný slot není nalezen, algoritmus se rekurzivně zavolá pro předchozí/následující rezervaci. Aby se předešlo zacyklení (např. u kruhových vazeb mezi plány), sleduje již navštívené rezervace pomocí množiny \texttt{visited}.

    \begin{lstlisting}
if (visited.has(item.PlanId)) {
    return {
	    PrevSuggestion: realPrev,
	    NextSuggestion: next,
    }
}
visited.add(item.PlanId)

// Recursive search for previous suggestion
if (prevSuggestion === null && item.Prev !== null) {
    const res = machineUsage
	.filter((y) => y.Plan.PlanId === item.Prev?.PlanId)
	.map((x: MachineUsage) => {
	    const result = mapRes(x)
	    return findPlanFull(result)
	})[0]
    if (res) {
	    const sug = determineClosestTime(res, visited)
	    prev = sug.PrevSuggestion
    }
}
// Same thing for next suggestion
    \end{lstlisting}
\end{enumerate}

Tento přístup kombinuje efektivitu (práce s již načtenými daty) s robustností (rekurzivní pokrytí všech možných kolizí). Zároveň respektuje kapacitu a možnost zakázání kolizí jiných rezervací. 
\section{Generování času pro trénink}

Poslední z funkcionalit systému je zároveň i ta nejkomplexnější. Nejprve systém musí data z databáze namapovat na potřebné typy pro grafy a následovně je potřeba tento graf vytvořit.

\subsection{Získání dat}
Navydory komplexitě řešení je databázový SQL dotaz vcelku jednoduchý, Jedná se o spojeníData získáme z databáze pomocí následujícího dotazu:

\begin{lstlisting}[language=SQL]
select * 
from plan_machine 
    inner join machine 
	on plan_machine.machine_id = machine.machine_id
    inner join reservation
	on plan_machine.plan_id = reservation.plan_id
WHERE reservation_time::date = input_reservation_date::date
AND start_time >= formattedStartTime::time
And machine.machine_id = ANY(machine_ids)
ORDER BY start_time, end_time;
\end{lstlisting}
Kdy input\_reservation\_date je vložené datum, formattedStartTime je formátovaný počáteční čas a machine\_ids je pole id přístrojů.
Žádné jiné podmínky není potřeba aplikovat, to by způsobilo chybějící záznamy, které pro vytvoření grafu budeme potřebovat.

\subsection{Tvorba dat}
Data podle kterých se bude graf vytvářet budou záviset na vstupních parametrech uživatele a získaných datech. Vstupní data nám reprezentuje parametr \texttt{collisions}, který je nepovinný parametr ve query HTTP dotazu. Pokud collisions není vloženo, tak kolize považujeme za povolené. Tento datový set budeme nazývat kolizním, v implementaci máme 2 typy, kolizní a nekolizní, tyto typy jsou definovány pomocí enumu \texttt{DataSetType}.

\begin{lstlisting}
enum DataSetType {
    COLLIDING,
    NON_COLLIDING,
}

let datasetType =
    collisions === true
	? DataSetType.COLLIDING
	: DataSetType.NON_COLLIDING

if (collisions === undefined) {
    datasetType = DataSetType.COLLIDING
}
\end{lstlisting}

Dále je systém data namapuje na jiný datový typ. Tímto typem je \texttt{NodeValue}.

\begin{lstlisting}
class NodeValue extends Model {
    machine: Machine
    start_time: Time
    end_time: Time
    reservation?: Reservation
    can_collide: boolean

    constructor(jsonData: IDictionary<any>) {
	    // ...
    }
}
\end{lstlisting}
Data systém převede a začne je upravovat pomocí funkce \texttt{CreateDataset}, tato funkce příjme parametry reprezentující data získána z databáze, jež převedena na do datového typu \texttt{NodeValue}, počáteční čas generování, počet lidí v rezervaci, a typ požadovaného datasetu ve formě Enum hodnoty \texttt{DataSetType}. Tato funkce vrátí požadovaná data, která následně budou využity pro tvorbu grafu. Výdledkem je pole polí, jež reprezentuje časová okna pro jednotlivé přístroje.


\begin{lstlisting}
async CreateDataset(
    input: NodeValue[],
    start_time: Time,
    amount_of_people: number,
    datasetType: DataSetType
) {
    try {
	    const data = this.prepareDataset(input, start_time)

	    if (datasetType === DataSetType.COLLIDING) {
		    return await GraphService.CreateGraphNodes(
			    this.getCollidingDataSet(data, amount_of_people)
		    )
	    }

	    return await GraphService.CreateGraphNodes(
		    this.getNonCollidingDataSet(data)
	    )
    } catch (err) {
	    throw err
    }
}
\end{lstlisting}

Výsledkem funkce getCollidingDataSet je profiltrované 

TODO

\chapter{Testy a CICD}
V softwarovém vývoji se lze potkat výhradně s dvěmi typy chyb. Jedná se o chyby syntatické a chyby logické. Syntatické chyby odchytává compiler, či interpreter při kompilování, nebo běhu programu. Jedná se o takovou systatickou analýzu. V dnešní době se tyto chyby dají odchytnout i dříve pomocí moderních editorů, jež využívají LSP. Syntatické chyby jsou tedy vcelku jednoduché na odchycení a lze se jich zbavit okamžitě. Logické chyby jsou v tomto ohledu mnohem horší. Namísto pomyslného boje proti compileru, nebo typové kontrole se jedná o boj proti sobě. Logické chyby se dají odchytit pouze užíváním výsledného produktu a zjištěním jiného stavu než je očekávaný. 

Řešení tohoto problému je jasné, po implementaci je potřeba program vyzkoušet a zjistit zda všechny potřebné komponenty fungují tak jak mají. Tento proces je ale velmi zdlouhavý a často opomenutý z několika důvodů, mezi které se zahrnuje jak lenost, tak vytíženost stejně jako změna priorit. Pro zjednodušení a zrychlení celého tohoto procesu bylo na bakalářské práci využita implementace Testování a CICD

\section{Testování}
Testy jsou pouze implementovány pro backendovou část bakalářské práce. Zpracování je díky pečlivému odchycení chyb a vracení správných Status kódů vymezena pouze na poslání dotazu na API a následné kontrole daného HTTP kódu. Testy jsou implementovány v programovacím jazyce python s pomocí knihovny pytest.

\section{CI/CD}
CI/CD představuje sérii kroků provedených automaticky při změnách v kódu. V této práci byla tato vývojová filozifie adaptována v obou formách pomocí github actions. viz Využité technologie.

\section{CD - Continuous deployment}
Co se kontinuálního nasazení kódu týče, tak řeší tvorbu Docker kontejneru. Tento kontejner je vytvořen jinak pro jiné účely. Pokud je půd pro spuštění automatického nasazení vyvolán pomocí vytvoření/přidání změn do pull requestu, tak se vytvoří a pošle tag s odpovídajícím číslem pull requestu. Pokud se jedná o sloučení do hlavní vývojové větve main, tak se jedná o tag latest.

\section{CI - Continuous integration}
V rámci kontinuální integrace se řeší spuštění testů pro api. Obecně jde o proces zajištění splnění funkčních i nefunkčních parametrů, jež program aplikuje. Mezi funkční parametry se zahrnuje samotná funkčnost kódu a splnění nároků implementované funkce. Nefunkční parametry zahrnují věci obecné ke kvalitě kódu, liting, formátování, a pod.

\chapter{Retrospektiva} \label{retrospective}

Tato bakalářská práce byla z mého pohledu velmi přínosná. Nicméně stejně jako každý vývoj se také tento potýkal s nejen pár problémy a to nejen implementačními, ale i určitými rozpory s ohledem na preference a technologická rozhodnutí. V této kapitole budou tyto překážky popsány společně se způsoby, jakými byly řešeny.

\section{Technologické Problémy}

Jeden z více osobních problémů, které nastaly, byla brzká technologická rozhodnutí. Důvody volby těchto technologií byly popsány v kapitole Využité technologie. Zde jsou popsány konkrétní překážky, jež byly spojovány s těmito rozhodnutími.

\subsection{Javascript}
Jak jsem již na začátku zmiňoval, tato technologická volba nese i značné nevýhody, jako například:

\begin{description}
\item[Dynamické datové typy] 
Dynamicky typované jazyky jsou náchylné na větší chybovost a menší kvalitu kódu \cite{pang2018programming}. V bakalářské práci se osvědčilo použití statických typů, které pomohly lépe definovat výstupy a vstupy.
\item[Chybějící vlastnosti jazyka] 
Jelikož JavaScript je dynamicky typovaný jazyk, chybí v něm koncepty jako například enum, interface a mnoho dalších, které například TS dodává \cite{typescriptlangHandbookEnums, typescriptlangHandbookInterfaces}.
\item[Řešení výjimek]
JS má oproti více moderním programovacím jazykům mě osobně nevyhovující systém odchytávání chyb. JS používá systém try/catch/finally, který se v moderních jazycích postupně nahrazuje konceptem Errors as values. Tato nevýhoda je pouze osobní preferencí, ale~tato preference je podmíněna. Pokud například mám určitou funkci, při jejímž průběhu může nastat chyba, tak v jazycích, které mají chybovatelné typy (například Result enum v Rustu), či funkce s více vracejícími hodnotami (jako například v Go), se jednoduše vrátí jiný výsledek. Díky tomuto přístupu je programátor nucen chybu ošetřit, aby se ujistil o existenci správného výsledku. To vede k robustnějšímu kódu. Pokud ale můžeme jednoduše výjimku vyhodit, čímž průběh programu narušíme a funkce stále vrací stejný typ, není přímo z definice této funkce jasné, zda může vyhodit výjimku. Jsme v tu chvíli nuceni tuto záležitost odchytávat při tom, jak na ni narazíme. To se stává problémem, když interagujeme s námi nenapsaným kódem.
\end{description}

Po několika měsících vývoje nastalo důležité technologické rozhodnutí ohledně přepsání celého backendu do jazyka TS. Toto rozhodnutí bylo klíčové pro další vývoj, jelikož vyřešilo dva ze tří problémů, které jsem s JavaScriptem měl.

TS a JS mají zároveň i jeden další problém, a tím je paralelismus se sdílenou pamětí. Na malé škále, jako je tato bakalářská práce, která nepocítila velké uživatelské zatížení, to není zřejmé, ale~myslím si, že by se aplikace lépe škálovala, kdybych backend napsal v jiném jazyce, jako je například Go.

\subsection{Implementace vlastního ORM}
I přesto, že jsem s momentální implementací ORM spokojený, zabrala velkou část vývoje. Tento čas by mohl být investován do jiných aspektů aplikace, které jsou popsány v možných rozšířeních práce. Tato časová investice nepřinesla až takové výhody, jaké jsem očekával. Bohužel na všechna rozšíření, jako programatické skládání dotazů, podmínek a migrační systém, čas nezbyl.

\chapter{Závěr}
TODO
I did it lol
\endinput


% Seznam literatury
\printbibliography[title={Literatura}, heading=bibintoc]

% Prilohy
\appendix
%\input{Chapters/Appendix1.tex}
%\input{Chapters/Appendix2.tex}

% Priloha vlozena primo do hlavniho LaTeX souboru. Ne vsechny prilohy je nutne mit ve zvlastnich souborech.
%\chapter{Dlouhý zdrojový kód}
%\lstinputlisting[label=src:CppExternal,caption={Dlouhý zdrojový kód v jazyce C++ načtený s externího souboru}]{SourceCodes/ArraySortingAlgorithms.cpp}

\end{document}
