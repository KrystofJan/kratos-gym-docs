Tato kapitola se zaměřuje na technologie využité při vývoji systému popsaného v této bakalářské práci. Jednotlivé technologie jsou zde představeny v kontextu jejich funkce v rámci využití napříč vrstvami v systému. Cílem této části je poskytnout čtenáři přehled o nástrojích a prostředích, které byly zvoleny, a zároveň objasnit důvody, proč byly konkrétní technologie preferovány před jinými.

Ústřední roli v celém ekosystému hraje JavaScript, který se stal jádrem celého vývojového procesu. Tato volba vycházela z mnoha později uvedených důvodů, ale i přesto, že odpovídá široce rozšířené praxi, byla klíčová pro volbu ostatních technologií.

Tato bakalářská práce prošla několika vývojovými iteracemi. V klíčových změnách týkajících se výběru technologií mezi těmito různými verzemi systému bude podrobně pojednáno v části nazvané změněné technologie. Konečný soubor technologií zahrnuje následující technologie: 
\begin{itemize}
    \item \textbf{Backend}
        \begin{itemize}
            \item PostgreSQL
            \item JavaScript/TypeScript
            \item Node.js
            \item Express.js
        \end{itemize}
    \item \textbf{Frontend}
        \begin{itemize}
            \item Vue.js
            \item Tailwind CSS
            \item ShadCN
            \item Clerk
            \item Zod
        \end{itemize}
    \item \textbf{Testy \& CI/CD}
        \begin{itemize}
            \item Docker
            \item Git / Github
            \item CI/CD - Github Actions
            \item Python \& pytest
        \end{itemize}    
\end{itemize}

V následujících podkapitolách jsou proto jednotlivé technologie detailně popsány včetně jejich výhod, nevýhod a konkrétního využití v rámci systému.

\section{JavaScript}
JavaScript je základním stavebním kamenem praktické implementace této bakalářské práce. Byla v něm i napsána úplně první verze celého backendu. V projektu byl JS použit z několika důvodů:

\begin{description}
  \item[Rozšířená komunita] \hfill \\ JS má obrovskou komunitu vývojářů, která nabízí bohatou podporu a nespočet open-source knihoven a frameworků, což usnadňuje a zrychluje vývoj \cite{dev52MDevelopers}.
  \item[Nativní podpora ve webových prohlížečích] \hfill \\ I v dnešní době, kdy máme mnohem více možností, je JS stále "Jazykem webu"\cite{ranjan2020javascript} a běží nativně ve všech moderních prohlížečích\cite{browserstackJavaScriptCode}, což z něj činí téměř nezbytnou součást vývoje webových aplikací. Pro jednoduchost a jednotu mezi vrstvami dávalo smysl používat jeden jazyk.
\end{description}

Použití JavaScriptu umožnilo rychlý start a flexibilní rozvoj projektu. Samozřejmě, že jeho volba s sebou přinesla i některé problémy a úskalí, se kterými se bylo potřeba v průběhu vývoje vypořádat. Tímto se budeme detailněji zabývat v kapitole \ref{retrospective}.

V pozdější části implementace došlo k rozhodnutí přejít na \texttt{TypeScript}, což byl důležitý krok směrem k lepší struktuře a stabilitě kódu.

\subsection{TypeScript}
Je to open-source programovací jazyk vyvíjený společností Microsoft \cite{typescriptlangDocumentationTypeScript}, který je postavený na JavaScriptu a rozšiřuje ho o typovou kontrolu\cite{goldberg2022learning}. TS poskytuje výhodu především ve větších projektech, kde je potřeba zajistit stabilitu a minimalizovat chyby spojené s dynamickým typováním JavaScriptu.

Statická typová kontrola v TS umožňuje odhalit chyby již během psaní kódu a nabízí větší jistotu při refactoringu \cite{enwiki:1258410189, sethi1996programming}. Přestože TS stále generuje běžný JavaScript, přináší strukturovanější a bezpečnější vývojový proces, což významně zlepšilo kvalitu kódu.

\section{PostgreSQL}
PostgreSQL je open-source objektově-relační databázový systém, který byl v této práci zvolen jako databázové řešení \cite{postgresqlPostgreSQL}. Tento databázový systém je známý svou stabilitou, rozšiřitelností a podporou pokročilých funkcí, jako jsou uložené procedury, transakce nebo plná podpora ACID (Atomicita, Konzistence, Izolace, Trvalost). PostgreSQL nabízí škálovatelnost a spolehlivost, které jsou klíčové pro náročné webové aplikace.

Prvotní verze byla vytvořena s použitím MySQL, protože jsem s touto technologií v minulosti již pracoval. Migrace na PostgreSQL proběhla kvůli možnosti využití bezplatného hostingu. Aplikace byla zpočátku vyvíjena na různých zařízeních a usiloval jsem o zachování datové konzistence bez nutnosti migračního systému. Nakonec tento požadavek vymizel a s ním i potřeba hostovat databázi mimo lokální prostředí. Nyní databáze běží v Docker kontejneru.

\section{Node.js}
Pro spuštění a chod programu potřebují programovací jazyky specifické běhové prostředí (Ang. Runtime environment)\cite{enwiki:1245152116, alfred2007compilers}. Při vývoji JavaScriptu byl vytvořen ``JavaScript engine'' pro webový prohlížeč Netscape Navigator. V podstatě se jednalo o velmi jednoduchý interpretor. Později se tento engine vyvinul v SpiderMonkey, který je dodnes používán v prohlížeči Mozilla Firefox \cite{newJavaScriptEngineModuleOwner}. Alternativa pro Chromium se nazývá V8\cite{v8JavaScriptEngine}.

Node.js poskytuje běhové prostředí pro JavaScript mimo prohlížeč. Vznikl v roce 2009 a umožnil použití JavaScriptu pro serverové aplikace. Je postaven na enginu V8 a díky tomu je velmi výkonný a efektivní.\cite{nodejsNodejsAbout}

Node.js je dnes největším běhovým prostředím pro JavaScript na světě. Kromě možnosti tvorby serverových aplikací poskytuje ekosystém balíčků spravovaný přes \texttt{npm}(Node Package Manager), což je největší repozitář softwarových balíčků na světě.\cite{npmjsAboutDocs}

Mezi alternativy patří například \texttt{Bun} a \texttt{Deno}, avšak \texttt{Node.js} stále představuje průmyslový standard, což je i důvod, proč byl zvolen pro tuto práci \cite{chhetri2016comparative}.

\section{Express.js}
Express.js je minimalistický webový framework pro Node.js, který byl použit pro tvorbu aplikačního programového rozhraní (API). Express usnadňuje práci s HTTP metodami a nabízí jednoduchou, ale flexibilní strukturu pro vývoj REST API\cite{expressjsExpressNodejs}.

Express umožňuje jednoduchou, ale transparentní práci s HTTP metodami, z něhož tvoří oblíbenou volbu pro vývoj webových aplikací všech velikostí. V této práci byl využit pro tvorbu REST API, které funguje jako komunikační most mezi frontendem a databází.

\section{Vue.js}
Vue.js je progresivní open-source framework pro tvorbu uživatelských rozhraní\cite{vuejsVuejs}. Byl zvolen pro frontend tohoto projektu díky své jednoduchosti a flexibilitě. Vue.js umožňuje rychlý vývoj komponent, které jsou zabaleny do velmi přehledné struktury.

Hlavní silnou stránkou Vue.js je jeho reaktivní systém správy stavu. Tento mechanismus automaticky sleduje závislosti mezi daty a rozhraním, což eliminuje potřebu explicitního řízení aktualizací \cite{vuejsVuejsDocs}.

Frameworky jako React používají imperativní správu stavu pomocí hooků (např. useState) \cite{bugl2019learn}, \texttt{Vue.js} umožňuje přímou mutaci reaktivních proměnných deklarovaných pomocí \texttt{ref()} nebo~\texttt{reactive()} \cite{vuejsVuejsDocs}. Tento rozdíl redukuje boilerplate kód a snižuje riziko chyb spojených s nesprávnou synchronizací stavu. Následující ukázka demonstruje implementaci čítače v Reactu s použitím hooku \texttt{useState}:
\begin{lstlisting}
function Example() {
  const [count, setCount] = useState(0);
  return (
    <div>
      <p>You clicked {count} times</p>
      <button onClick={() => setCount(count + 1)}>
        Click me
      </button>
    </div>

  );
\end{lstlisting}
Zde je nutné explicitně deklarovat jak stavovou proměnnou (count), tak odpovídající setter (setCount). Každá změna stavu vyžaduje volání setter funkce, což vede k přerenderování komponenty. Ačkoli tento model poskytuje granularitu kontroly, vyžaduje od vývojáře důsledné oddělení stavové logiky a prezentace. Naproti tomu stejná funkcionalita ve Vue.js využívá reaktivní referenci:
\begin{lstlisting}
<script setup>
import { ref } from "vue";
const counter = ref(0);
</script>

<template>
  <div>
    <p>You clicked {{ counter }} times</p>
    <button @click="counter++">Click me</button>
  </div>
</template>
\end{lstlisting}
Proměnná \texttt{counter} je obalena proxy objektem, který automaticky detekuje změny prostřednictvím getterů a setterů generovaných funkcí \texttt{ref()}. Mutace hodnoty přímým přiřazením (counter++) spouští reaktivní aktualizaci DOM bez nutnosti manuálního volání setterů.

\section{Tailwind CSS}
Tailwind je framework pro CSS, který výrazně zrychluje stylování uživatelského rozhraní \cite{tailwindcssTailwindRapidly}. Namísto tradičních předpřipravených komponent nabízí Tailwind množství utility tříd, které lze kombinovat a vytvářet tak design přímo v HTML komponentách.

Použití Tailwindu v tomto projektu umožnilo vytvořit flexibilní a rychle upravitelný design bez nutnosti psát samotné CSS.

\section{Shadcn}
Shadcn je sada komponent výhradně pro React a Tailwind, která zjednodušuje integraci uživatelského rozhraní do aplikací. Umožňuje rychlé nasazení a styling komponent pomocí Tailwind CSS, čímž snižuje čas potřebný na vývoj frontendu. Jedná se o předem nastylované Radix komponenty \cite{shadcnIntroduction}.

Na rozdíl od ostatních sad komponent, jako například Bootstrap, shadcn přidá kód komponent přímo do projektu, tudíž jsou plně upravitelné. Vue komunita vytvořila svou neoficiální verzi Shadcn s názvem Shadcn-vue, která byla v této práci využita \url{https://www.shadcn-vue.com/}.

Použití této knihovny komponent ulehčilo vývoj webového uživatelského rozhraní s dodáním celistvého vzhledu.

\section{Clerk}
Clerk je autentizační a správcovská služba uživatelů, která zjednodušuje implementaci přihlašování, registrací a správy uživatelů v aplikacích. Nabízí snadnou integraci s frontendovými technologiemi, jako je Vue.js\cite{clerkClerkAuthentication}. 

V tomto projektu je Clerk použit pro autentizaci, což zajišťuje bezpečný přístup k aplikaci.

\section{Zod}
Zod je knihovna pro validaci dat. Umožňuje definovat schémata, která popisují tvar dat, a následně je validovat \cite{zodTypeScriptfirstSchema}.

V tomto projektu je Zod využíván k validaci dat přicházejících ze shadcn formulářů. Tímto zajišťuje konzistenci a správnost dat na serverové straně.

\section{Docker}
Docker je nástroj, který slouží k vývoji, distribuci a spouštění aplikací. Umožňuje oddělit aplikace od infrastruktury, což usnadňuje a zrychluje nasazování softwaru. Díky Dockeru se dá infrastruktura spravovat podobně jako aplikace. Pomocí metod, které Docker nabízí pro testování, nasazování a~distribuci kódu, se dá zkrátit čas mezi napsáním kódu a jeho spuštěním v ostrém provozu\cite{a2024_what}.

V rámci tohoto projektu Docker poskytuje efektivní a rychlé řešení, které umožňuje vývoj a~distribuci aplikace napříč různými platformami.

\section{Git / Github}
Git je distribuovaný systém pro správu verzí, který umožňuje efektivní sledování změn a správu různých verzí souborů. Je široce využíván díky své flexibilitě a podpoře týmové spolupráce, čímž se stal standardem mezi verzovacími systémy \cite{spinellis2012git}.

V této bakalářské práci byl Git nasazen od počátku jako klíčový nástroj pro verzování. Pro sdílení a správu repozitáře byl zvolen GitHub, populární platforma rozšiřující Git o funkce pro ukládání, správu a spolupráci v cloudu. Díky svému intuitivnímu prostředí se GitHub stal oblíbeným řešením pro projekty všech velikostí \cite{githubAboutGitHub}.

\section{CI/CD - Github actions}
GitHub Actions je platforma pro kontinuální integraci a doručování (CI/CD), která umožňuje automatizovat procesy, jako jsou sestavení, testování a nasazování aplikací. S GitHub Actions lze vytvářet pracovní postupy, které automaticky sestaví a otestují každý pull request v repozitáři, nebo nasadí sloučené pull requesty přímo do produkce\cite{githubinc_2024_understanding}.

Při vývoji rezervačního systému byly GitHub Actions především použity pro vytvoření Docker kontejneru a pro spuštění testů.

\section{Python a pytest}
Python je vysoce úrovňový interpretovaný skriptovací jazyk, který se vyznačuje unikátním způsobem strukturování kódu. Namísto tradičního používání složených závorek pro seskupování bloků kódu využívá Python bílých znaků (whitespace), což přispívá k jeho čitelnosti a jednoduchosti \cite{kuhlman2009python, van1990functional}.

V rámci tohoto projektu je Python využit pro vývoj API testů, přičemž byl využit framework pytest. Pytest je vcelku jednoduchý a s kombinací s možností rychlého vývoje v Pythonu toto technologické rozhodnutí umožnilo opravdu rychlý vývoj jednoduchých testů.
