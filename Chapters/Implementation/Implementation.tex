\chapter{Implementace}
V kontextu již definovaného implementačního rámce (technologický stack, datové toky, architektura) přistupujeme k analýze klíčových algoritmických komponent systému: doporučování časů přístrojů, generování časů pro tréninky způsoby a řešení kolizí.

\section{Doporučení přístrojů}
Jak již bylo popsáno v kapitole \textbf{Teoretická analýza}, tak koncept doporučení přístrojů není příliž složitý. Co se praktické implementace týče, tak je to obdobné.

\subsection{Frontend}
Nejprve je potřeba aby uživatel přístroje vybral. K tomu poslouží řada checkboxů, které tyto přístroje budou reprezentovat. Checkboxy mohou působit jako ne příliš dobré řešení, ale pro demostrační účely této bakalářské práce to bude stačit.

Ve výsledku budou přístroje uloženy v poli. Toto pole bude monitorováno pomocí computed property, která v sobě bude uchovávat informaci ohledně nejčastějším id kategorie nalezeným v těhto přístrojích.

Computed properties (česky: Vypočítané vlastnosti) jsou vlastnosti, které jsou vypočtené z jiných reakticvích proměnných. Jejich největší výhoda spočívá v sebekontrolování jejich závislostí, t.j. jakmile se jakákoli reaktivní proměnná, na které je vypočítáná vlastnost závislá změní, hodnota této vlastnosti je přepočítá. Zde tato vlastnost nabývá podoby proměnné mostFrequentCategoryId. Computed property je definována pomocí definice getter funkce. Tato computed property bude sloužit při následnému volání na API, které se uskuteční pomocí watcheru.

Podobně jako u computed vlastností je watch způsob jakým lze reagovat na změnu reaktivní proměnné, ale s tím rozdílem, že watcher nedrží žádnou hodnotu. Watcher pouze sleduje reaktivní proměnnou a při její změně zavolá callback funkci. Watcher by se dal jednoduše vysvětlit jako event handler pro reaktivní proměnné ve Vue.js.

\begin{lstlisting}
const selectedMachines = ref<Machine[]>([])
const recommendMachines = ref<Machine[]>([])

const mostFrequentCategoryId = computed(() => {
  // Find the and return the categoryId with the most occurences
});

watch(
  mostFrequentCategoryId,
  async (newId) => {
    // Get the recommended machines from the API and assign them to recommendedMachines
  },
  { deep: true }
)
\end{lstlisting}

Kombinací těchto dvou konceptů docílíme menší frekvenci dotazů na API, čož umožnuje menší zatížení backendu a menší potřebu překreslovat doporučené přístroje na Frontendu, což by bylo způsobeno neustálým přepisováním hodnot reaktivních proměnných, jež jsou v template sekci volány.

\subsection{Backend}
Ze strany backendu se jedná o dotaz na databázi. Tento dotaz je poněkud komplikovanější. Má více vstupních parametrů a zároveň jeho implementace požaduje vnořené příkazy. Jak již bylo zmíněno v popisu databáze, tak pro tyto případy bylo učiněno rozhodnutí takovéto příkazy zaobalit do procedury přímo v databázi. V implementaci API se tedy volá pouze tato procedura.

\begin{lstlisting}
const result: Machine[] = await this.sql<Machine[]>`
    SELECT * 
    FROM get_machines_in_same_category(${id})
`
\end{lstlisting}

Volaná procedura tedy obsahuje pouze následující SQL dotaz. Tento dotaz nám propojuje 3 tabulky. Machine (reprezentuje individuální přístroj, např. Bench, Squat rack, atd.), machine exercise type (reprezentuje N:M vztah mezi cvikem a přístrojem) a exercise type (reprezentuje cvik, např. Dřep, Biceps curl, atd.). Výsledek spojených tabulek je následně profiltrován pomocí vnořeného SQL dotazu, který vyhledává první id kategorie, které přístroj s vloženým id má. Následně se z výsledku předešlé filtrace vyjme přístroj s id v parametru procedury. Nazávěr se data sestupně seřadí podle jejich popularity.

\begin{lstlisting}[language=SQL]
SELECT DISTINCT m.*
FROM machine m
JOIN machine_exercise_type met ON m.machine_id = met.machine_id
JOIN exercise_type et ON met.exercise_type_id = et.exercise_type_id
WHERE et.category_id = (
    SELECT et.category_id
    FROM machine_exercise_type met
    JOIN exercise_type et ON met.exercise_type_id = et.exercise_type_id
    WHERE met.machine_id = input_machine_id
    LIMIT 1
)
AND m.machine_id != input_machine_id
ORDER BY m.popularity_score DESC;
\end{lstlisting}
Přirozeně se o tomto řešení dá říci, že je zbytečně komplikované a jelikož hledáme nejčastější kategorii na straně frontendu. Toto sebou ale nese jeden problém. Konkrétně se jedná o ten fakt, že doporučení pomocí přístroje nám nabízí více možností z ohledu dalšího rozšíření systému. Dále může jeden přístroj mít více kategorií, tudíž nemůžeme vědět jakou konkrétně chceme najít. Hledání kategorie sloužilo pouze pro vymezení možností v rámci tréninku a ne v rámci individuálních přístrojů.

\section{Doporučení najbližšího dostupného času}

Pro doporučení nejbližšího dostupného času se musí vzít v potaz kolize s jinými rezervacemi, což je problematika, která byla popsaná v teoretické analýze. Co popsáno ale nebylo je způsob jak tuto kolizi nalezneme v reálném čase, t.j. ve chvíli, kdy definujeme časové okno.

\subsection{Hledání kolizí v reálném čase}
Tuto část rozebíráme v praktické implementaci jelikož je řešení velmi úzce spojené s praktickou implementací na frontendu. 

Jako jednoduché řešení se nabízí po každé změně časového okna poslat dotaz skrze API, který by vrátil počet rezervací s vypnutými kolizemi v daný čas a pokud by byl počet větší, nebo roven jedné, tak bychom časové okno označili jako kolidující. Tento způsob by ale způsobil obrovské množství dotazů na API, což by způsobilo vysoké zatížení backendu. To by se částečně dalo vyřešít využitím WebSockets, nebo podobnými způsoby.

Ačkoli pokusy o funkčnost výše zmíněného řešení jsou lákavé v ohledu na fascinaci s jinými řešeními, nabízí se také mnohem intuitivnější řešení. Při pokročení na krok konfigurace časového okna systém syšle dotaz na backend o aktuální rezervace pro daný den, pro každý vybraný přístroj. Tyto data budou následně používány při validaci jednotlivých časových oken pomocí zod validačních schémat. Pokud systém kolizi při validaci najde, tak zašle dotaz pro vyhledání nejbližších časů pro daný přístroj.


\subsection{Backend}
Na backendu algoritmus pracuje s daty získanými z databáze, která obsahují nejen plány rezervací pro daný přístroj, ale také informace o časových oknech předchozích/následujících rezervací a kapacitních omezeních. Cílem je najít nejbližší volný interval stejné délky jako požadované časové okno, a to buď před první kolidující rezervací, nebo za poslední kolidující rezervací, s ohledem na možnost povolení kolizí.

Data jsou získávána pomocí následujícího sql dotazu:
\begin{lstlisting}[language=SQL]
SELECT pm.*,
  CASE
    WHEN
      ((pm.previous_start_time <= pm.end_time AND pm.previous_end_time >= pm.start_time) OR
      (pm.previous_start_time <= pm.next_end_time AND pm.previous_end_time >= pm.next_start_time) OR
        (pm.start_time <= pm.next_end_time AND pm.end_time >= pm.next_start_time))
      AND m.max_people >= SUM(r.amount_of_people) + input_amount_of_people
  THEN true
  ELSE false
END AS can_fit
FROM get_plan_machines_with_next_and_prev(input_machine_id, input_reservation_date) pm
  INNER JOIN reservation r ON r.plan_id = pm.plan_id
  INNER JOIN machine m ON m.machine_id = pm.machine_id
GROUP BY
  pm.plan_id, pm.machine_id, pm.can_disturb, pm.start_time, pm.end_time,
  pm.previous_plan_id, pm.previous_start_time, pm.previous_end_time,
  pm.next_plan_id, pm.next_start_time, pm.next_end_time,
  r.amount_of_people, m.max_people;
END;
\end{lstlisting}
Jak si lze všimnout tento SQL dotaz vyhledává výsledky jiné procedury. Vráceným vásledkem této procedury je každá rezervace společně s časovými okny předchozí a následující rezervace.
Dotaz také zjistí, zda se počet lidí ve vytvářené rezervaci vleze do kolidovaných rezervací s ohledem na kapacitu daného přístroje. Výsledek těhto podmínek je booleanová hodnota can\_fit

Dalším krokem po získání dat z databáze je aplikovat algoritmus implementovaný v metodě \texttt{SuggestTimes}, která se nachází ve třídě \texttt{MachineService}. Tento algoritmus následuje tyto kroky:

\begin{lstlisting}
static async SuggestTimes(
    machineUsage: MachineUsage[],
    desiredTimeRange: TimeRange,
    canDisturb: boolean
): Promise<Suggestion> {
    ...
}
\end{lstlisting}

Tato funkce příjmá výsledek dotazu na databázi v podobě pole v proměnné \texttt{machineUsage}, \texttt{desiredTimeRange} reprezentující žádáné časové okno a \texttt{canDisturb}, jež značí povolení kolizí. Metoda vrací typ \texttt{Suggestion}.

\texttt{Sugestion} obsahuje 2 časová okna, jedno před kolidujícím a druhé za ním. Typ \texttt{TimeRange} reprentuje časové okno samotné a tvoří jej 2 časové záznamy (Start, End) a booleanová hodnota značící existenci kolize.

\begin{lstlisting}
interface TimeRange {
    StartTime: Time
    EndTime: Time
    isColiding: boolean
}

interface Suggestion {
    PrevSuggestion: TimeRange
    NextSuggestion: TimeRange
}
\end{lstlisting}
\begin{enumerate}
    \item \textbf{Detekce kolizí}
	Požadované časové okno a všechny rezervace se převedou na minutovou reprezentaci (např. 14:30 → 870 minut). Kolize se detekuje pomocí intervalové logiky:
\[
K \exists \iff \max(t_{1start}, t_{2start}) \leq \min(t_{1end}, t_{2end})
\]
	Kdy \( t_{1} \) je čas pro danou rezervaci a \( t_{2} \) je čas pro žádané časové okno.
	Výsledkem je seznam \texttt{colidingPlan} s kolidujícími rezervacemi.
	
	V kódu tato část vypadá nějak takto:
	
    \begin{lstlisting}
const colidingPlan = machineUsage
	.map((x: MachineUsage) => {
	    const result = mapRes(x)
	    return findPlanFull(result)
	})
	.find((time) => {
	    const {StartTime, EndTime}  = time
	    const endTimeInMinutes = EndTime.hour * 60 + EndTime.minute
	    const startTimeInMinutes = StartTime.hour * 60 + StartTime.minute
	    const minEndTime =
		    endTimeInMinutes <= desiredEndInt ? endTimeInMinutes : desiredEndInt
	    const maxStartTime =
		    startTimeInMinutes >= desiredStartInt ? startInt : startTimeInMinutes
	    return maxStartTime <= minEndTime
	})
    \end{lstlisting}

    \item \textbf{Výpočet potenciálních časových oken}
    \begin{itemize}
    	\item Před kolidující rezervaci: Pokud mezi koncem předchozí rezervace a začátkem aktuální rezervace existuje mezera větší nebo rovna délce požadovaného okna, vypočítá se okno jako \texttt{startTimeInMinutes - duration}.
    	\item Po kolidující rezervaci: Pokud mezi koncem aktuální rezervace a začátkem následující rezervace existuje dostatečná mezera, vypočítá se okno jako \texttt{endTimeInMinutes}. 
	

    \begin{lstlisting}
// Previous suggestion
if (prevEndTimeInMinutes === null || prevEndTimeInMinutes + duration <= startTimeInMinutes) {
    const startSuggest = startTimeInMinutes - duration
    prevSuggestion = createSuggestion(startSuggest, false)
    prev = prevSuggestion
}

// Next suggestion
if (nextStartTimeInMinutes === null || nextStartTimeInMinutes - duration >= endTimeInMinutes) {
    const startSuggest = endTimeInMinutes
    nextSuggestion = createSuggestion(startSuggest, false)
    next = nextSuggestion
}
    \end{lstlisting}
    \end{itemize}

    \item \textbf{Rekurzivní průchod}
    Pokud žádný slot není nalezen, algoritmus se rekurzivně zavolá pro předchozí/následující rezervaci. Aby se předešlo zacyklení (např. u kruhových vazeb mezi plány), sleduje již navštívené rezervace pomocí množiny \texttt{visited}.

    \begin{lstlisting}
if (visited.has(item.PlanId)) {
    return {
	    PrevSuggestion: realPrev,
	    NextSuggestion: next,
    }
}
visited.add(item.PlanId)

// Recursive search for previous suggestion
if (prevSuggestion === null && item.Prev !== null) {
    const res = machineUsage
	.filter((y) => y.Plan.PlanId === item.Prev?.PlanId)
	.map((x: MachineUsage) => {
	    const result = mapRes(x)
	    return findPlanFull(result)
	})[0]
    if (res) {
	    const sug = determineClosestTime(res, visited)
	    prev = sug.PrevSuggestion
    }
}
// Same thing for next suggestion
    \end{lstlisting}
\end{enumerate}

Tento přístup kombinuje efektivitu (práce s již načtenými daty) s robustností (rekurzivní pokrytí všech možných kolizí). Zároveň respektuje kapacitu a možnost zakázání kolizí jiných rezervací.

\section{Generování času pro trénink}
Jednou z nejobtížnějších funkcí systému je ta poslední. Nejprve je nutné, aby systém přetypoval data z databáze pro účely grafů a poté došlo k vytvoření daného grafu.

\subsection{Frontend}
Frontend implementace zahrnuje pouze API dotaz pro získání generovaných dat na straně klienta, nastavení hodnot pro vstupní prvky, a přidání tlačítka určeného pro možnost "Opětovného generování" sekvence s časovými okny. Toto opětovné generování nebude opětovně odesílat API dotaz, ale zvolí jiný z navrácených průchodů.

\subsection{Získání dat}
Napříč složitostí problému je SQL dotaz pro databázi relativně přímočarý. Dotaz zahrnuje spojování tabulek přístroje (machine) a rezervace přes jejich asociaci s plánem rezervace. Navíc dotaz využívá podmínkové ošetření a řazení dat podle času. Data z databáze obdržíme prostřednictvím následujícího dotazu:

\begin{lstlisting}[language=SQL]
select * 
from plan_machine 
    inner join machine 
	on plan_machine.machine_id = machine.machine_id
    inner join reservation
	on plan_machine.plan_id = reservation.plan_id
WHERE reservation_time::date = input_reservation_date::date
AND start_time >= formattedStartTime::time
And machine.machine_id = ANY(machine_ids)
ORDER BY start_time, end_time;
\end{lstlisting}
Kdy je v rámci input\_reservation\_date zadané datum, formattedStartTime představuje formátovaný začátek času a machine\_ids obsahuje pole identifikátorů strojů. Není nutné aplikovat žádné další podmínky, neboť by to mohlo vést k chybějícím záznamům, které jsou nezbytné pro generování grafu.

\subsection{Tvorba dat}
Údaje, na základě kterých bude graf sestaven, budou záviset na uživatelských vstupních parametrech a informacích získaných z databáze. Vstupní data jsou reprezentována parametrem \texttt{collisions}, který je nepovinným parametrem v dotazu HTTP. Pokud parametr collisions není zadán, považujeme kolize za povolené. Tento soubor dat nazveme kolizním a při implementaci rozlišujeme 2 typy: kolizní a nekolizní. Tyto typy jsou určeny prostřednictvím enumu.\texttt{DataSetType}.

\begin{lstlisting}
enum DataSetType {
    COLLIDING,
    NON_COLLIDING,
}

let datasetType =
    collisions === true
	? DataSetType.COLLIDING
	: DataSetType.NON_COLLIDING

if (collisions === undefined) {
    datasetType = DataSetType.COLLIDING
}
\end{lstlisting}

Následně systém převádí data na jiný datový typ, který je označen jako \texttt{NodeValue}.

\begin{lstlisting}
class NodeValue extends Model {
    machine: Machine
    start_time: Time
    end_time: Time
    reservation?: Reservation
    can_collide: boolean

    constructor(jsonData: IDictionary<any>) {
	    // ...
    }
}
\end{lstlisting}

Funkce \texttt{CreateDataset} zpracovává data tím, že přijímá parametry získané z databáze a jejich převedením na datový typ \texttt{NodeValue}. Tyto parametry zahrnují počáteční čas generování, počet osob v rezervaci a typ datasetu, který je reprezentován hodnotou Enum \texttt{DataSetType}. Výstupem této funkce je generovaný graf, který se vytváří pomocí funkce CreateGraphNodes.

CreateGraphNodes upravuje dataset založený na zadaných vstupních datech. Pomocí funkce prepareData se pro každý přístroj přidají prázdná časová okna. Funkce getCollidingDataSet a getNonCollidingDataSet se používají k transformaci a filtrování dat. Funkce getCollidingDataset odstraní záznamy, do nichž se zadaný počet lidí v rezervaci již nevejde, dále propojuje tréninky s navazujícími prázdnými intervaly. Naopak, getNonCollidingDataSet poskytuje data zcela bez rezervací. Výsledkem je pole grafových uzlů se stávajícími hranami. Tyto hrany se vytvářejí mezi všemi možnými sousedními uzly tak, že pokud je konečný čas současného uzlu menší nebo roven konečnému času jiného uzlu.

\begin{lstlisting}
async CreateDataset(
    input: NodeValue[],
    start_time: Time,
    amount_of_people: number,
    datasetType: DataSetType
) {
    try {
	    const data = this.prepareDataset(input, start_time)

	    if (datasetType === DataSetType.COLLIDING) {
		    return await GraphService.CreateGraphNodes(
			    this.getCollidingDataSet(data, amount_of_people)
		    )
	    }

	    return await GraphService.CreateGraphNodes(
		    this.getNonCollidingDataSet(data)
	    )
    } catch (err) {
	    throw err
    }
}
\end{lstlisting}

\subsection{Průchod grafem}

Nejprve se vytvoří instance grafu. Graf má vrcholy seřazené podle času. Třída grafu zahrnuje metodu findAllPaths, která identifikuje všechny potenciální cesty grafem. Tyto cesty začínají v libovolném počátečním vrcholu definovaném konkrétním zařízením. Systém bude procházet těmito zařízeními a pro každé z nich nalezne cestu. Funkce walk se volá rekurzivně pro každý aktuální přístroj, který dosud nebyl navštíven, dokud cesta nezahrnuje všechny vrcholy s přístroji. Je nutné postupně kontrolovat navštívené vrcholy. Pomocí nich bude systém vědět, jaké přístroje již navštívil. Prakticky v kódu to vypadá nějak takto:

\begin{lstlisting}
walk(
    currentNode: GraphNode,
    visited: number[] = [],
    result: Path = new Path(this.desiredMachines)
): Path {
    visited.push(currentNode.node_id)
    const neighbors = currentNode.neighbors.filter(
	(node) => !visited.includes(node.node_id)
    )

    const visitedMachines = result.nodes.map(
	(x) => x.value.machine.MachineId
    )

    for (const neighbor of neighbors) {
	if (visited.includes(neighbor.node_id)) {
	    continue
	}
	const newTime = TimeUtils.addTime(
	    this.currentTime,
	    neighbor.value.machine.AvgTimeTaken
	)

	if (
	    TimeUtils.canFitInTime(
		[neighbor.value.start_time, neighbor.value.end_time],
		newTime
	    ) &&
	    !visitedMachines.includes(neighbor.value.machine.MachineId)
	) {
	    result.addNode(
		new GraphNode(
		    new NodeValue({
			start_time: this.currentTime,
			end_time: newTime,
			machine: neighbor.value.machine,
			reservation: neighbor.value.reservation,
		    }),
		    neighbors,
		    neighbor.node_id
		)
	    )
	    this.currentTime = newTime
	    if (result.isFound()) {
		break
	    }
	    return this.walk(neighbor, visited, result)
	}
    }
    return result
}
\end{lstlisting}

Jak lze vidět, jak rekurzivní volání provádíme pouze pokud je možné společně s průměrným časem přístroje se vlést do časového okna sousedícího vrcholu. Postupně měníme současný čas , jež v danou chvíli máme, pro udržení kontextu rezervace. Výsledky průchodu následně vrátíme na klienta.

Výsledkem kombinace všech těchto metod je pole několika průchodů grafem s měnícím se začátečním přístrojem. Tyto průchody sice vždy trvají stejně, ale s jiným počátkem (například o 15 minut později). Tímto získáme pro danou chvíli všechny možné optimální způsoby, jakými lze trénink sestavit. Tento výsledek následně vrátíme na frontend, kde se hodnoty uloží do dále konfigurovatelných časových oken, jež jsou využívána v manuální konfiguraci.