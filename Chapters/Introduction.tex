\chapter{Úvod}
\label{sec:Introduction}
V současné době převažují ve fitness komunitě rezervační systémy odlišné od návrhu představeného v této bakalářské práci. Stávající řešení se primárně zaměřují na rezervaci osobních tréninků s certifikovanými trenéry nebo na alokaci prostoru pro skupinové aktivity, jako jsou spinningové lekce, boxerské tréninky či jiné specializované kolektivní programy. Tyto systémy však nedostatečně řeší problém přetížení fitness center, který se výrazně projevuje zejména v období zvýšené poptávky (např. na začátku roku). Následkem toho jsou návštěvníci nuceni průběžně upravovat své tréninkové plány podle aktuální dostupnosti přístrojů a kapacity zařízení, což snižuje jak efektivitu tréninku, tak celkovou spokojenost klientů.

Hlavní limitace současných systémů tkví v neschopnosti dynamicky reagovat na vytíženost jednotlivých přístrojů. Místo toho, aby monitorovaly stav zařízení v reálném čase (např. obsazenost běžeckých pásů, činek nebo strojů). To vede k nerovnoměrnému vytížení vybavení, častým časovým překryvům a nevyužitým kapacitám v méně frekventovaných časech.

Tento problém je mimořádně komplexní, protože je nutné zohlednit nejen maximální kapacitu posilovny, ale také vyhnout se kolizím v používání jednotlivých přístrojů. Na trhu i v open-source komunitě momentálně neexistuje žádné podobné řešení, které by tento problém řešilo efektivním a škálovatelným způsobem.

Tato bakalářská práce se zaměřuje na návrh a implementaci rezervačního systému pro posilovnu, který inteligentně doporučuje vhodné časy a přístroje pro jednotlivé uživatele. Výsledkem je REST API s webovou aplikací, která umožňuje uživatelům intuitivně spravovat své rezervace a získávat optimální doporučení pro jejich tréninkové plány.
\endinput