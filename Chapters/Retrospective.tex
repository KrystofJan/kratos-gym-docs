\chapter{Retrospektiva}
\subsection{Problémy}
\subsubsection{Javascript}
Jak jsem již na začátku zmiňoval, tak tato technologická volba nese i značné nevýhody jako například

\begin{description}
  \item[Dynamické datové typy] \hfill \\ Dynamicky typované jazyky jsou náchylné na větší chybovost a menší kvalitu kódu \cite{pang2018programming}. V bakalářské práci se osvědčilo použití statických typů, které pomohly lépe definovat vysupy a výstupy.
  % TODO upravit%
  \item[Chybějící vlastnosti jazyka] \hfill \\ Jelikož JavaScript je dynamicky typovaný jazyk, tak v něm chybí koncepty jako například enum, interface a mnoho dalších, které například TypeScript dodává.
\end{description}

\subsection{Získané zkušenosti}

\subsection{Možné rozšíření práce}