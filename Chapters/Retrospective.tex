\chapter{Retrospektiva}

Tato bakalářská práce byla z mého pohledu velmi přínosná. Nicméně stejně jako každý vývoj se i tento potýkal s nejen pár problémy a to ne jen implementačními, ale i určitými rozpory s ohledem na preference a technologickýmé rozhodnutí. V této kapitole budou tyto překážky popsány společně se způsoby jakými byly řešeny.

\section{Technologické Problémy}

Jeden z více osobních problémů, které nastaly byly s brzkými technologickými rozhodnutími. Důvody volby těhto technologíí byly popsány v kapitole Využité technologie. Zde jsou popsány konkrétní překážky, jež byly spojovány s těmito rozhodnutími.

\subsection{Javascript}
Jak jsem již na začátku zmiňoval, tak tato technologická volba nese i značné nevýhody jako například

\begin{description}
  \item[Dynamické datové typy] \hfill \\ Dynamicky typované jazyky jsou náchylné na větší chybovost a menší kvalitu kódu \cite{pang2018programming}. V bakalářské práci se osvědčilo použití statických typů, které pomohly lépe definovat vysupy a výstupy.
  \item[Chybějící vlastnosti jazyka] \hfill \\ Jelikož JavaScript je dynamicky typovaný jazyk, tak v něm chybí koncepty jako například enum, interface a mnoho dalších, které například TypeScript dodává.
  \item[Řešení výjimek] \hfill \\ Javascript má oproti více moderním programovacím jazykům mě osobně nevyhovující systém odchytávání chyb. Javascript používá systém try/catch/finally, který se v moderních jazycích postupně nahrazuje konceptem Errors as values. Tato nevýhoda je pouze osobní preferencí, ale tato preference je podmíněna. Pokud například mám určitou funkci, při jejímž průběhu může nastat chyba, tak v jazycích, které mají chybovatelné typy jako (například Result enum v rustu), či funkce s více vracejicími hodnotami (Jako například v Go), se jednoduše vrátí jiný výsledek. Díky tomuto přístupu je programátor nucen chybu ošetřit aby se ujistil o existenci správného výsledku. To vede k více robustnějšímu kódu. Pokud ale můžeme jednoduše výjimku vyhodit, čímž průběh programu narušíme a funkce stále vrací stejný typ. Není přímo z definice této funkce jasné, zda může vyhodit výjimku. Jsme v tu chvíli nuceni tuto záležitost odchytávat při tom, jak na ně narazíme. To se stává problémem, když interagujeme s námi nenapsaným kódem.
\end{description}

Po několika měsících vývoje nastalo důležité technologické rozhodnutí ohledně přepsání celého backendu do jazyka TypeScript. Toto toto rozhodnutí bylo klíčové pro další vývoj, jelikož vyřešil dva ze tří problémů, které jsem s javascriptem měl.

Typescript a JavaScript mají zároveň i jeden další problém a tím je paralelizmus se sdílenou pamětí. Na malé škále jako je tato balalářská práce, která nepocítíla velké uživatelské zatížení, to nelze úplně vidět, ale myslím si, že by se aplikace lépe škálovala, kdybych backend napsal v jiném jazyce jako je například GO.

\subsection{Implementace vlastního ORM}

I přesto, že jsem s momentální implementací ORM spokojený, tak zabrala velkou část vývoje. Tento čas by mohl být investován na jiné aspekty aplikace, které jsou popsány v možných rozšíření práce. Tato časová investice nepřinesla až takové výhody jaké jsem očekával. Bohužel na všechna rozšíření jako programatické skládání dotazů, podmínek a migrační systém čas nezbyl.

\subsection{Získané zkušenosti}

Celkově si troufám říci, že jsem přes tento proces nabyl spoustu zkušeností, které jsem schopen aplikovat v praxi. Jsem nyní více sebevědomý ve svých schopnostech. Ustanovil jsem si určité preference, jako například větší obliba striktně typovaných kompilovaných jazyků. Oprášil jsem si své frontend dovednosti, jež jsem od studia střední školy mimo stylování nepoužil. Dále jsem se samozřejmě zlepšil v mých schopnostech psát JavaScript a TypeScript.

\subsection{Možné rozšíření práce}

Zde můžete nalést seznam dodatků, které bych rád do práce dále přidal.

\begin{enumerate}
	\item Kreditový systém - Každá rezervace by byla za určitý počet kreditů, tyto kredity by se odvíjely od skutečných peněz. Zákazník by mohl například dostávat i věrnostní kredity a podobně.


	\item Napojení platící služby - Kreditový systém by potřeboval i platící službu jako například stripe, jež by umožnila zasílat platby přes internet.

	\item Statistiky - Rád bych do administrace přidal statistyky chodu posilovny. Například nejúžívanější přístroje a pod.

	\item Mapa posilovny - Konfigurátor jsem původně chtěl koncepovat v podobě interaktivní mapy posilovny. Toto řešení bylo ale nakonec zaměněno za checkboxy a konfiguraci časových oken. Rád bych ale tento způsob proskoumal

	\item Uložení tréninkového plánu - Rád bych také chtěl implementovat předpřipravéné tréninkové plány. V databázi pro toto již existují potřebné tabulky, ale k implementaci jsem se nedostal

	\item Možnost generace tréninku s "dírami" - Jako díry si můžeme představit volný čas mezi časovými okny zvolenými pro přístroj. Je to chvíle, kdy zákazník v rezervaci nemá zádný přístroj zarezervovaný. Generování těhto tréninků by mohlo teoreticky nést výsledek více efektivního zabrání přístrojů.
\end{enumerate}
