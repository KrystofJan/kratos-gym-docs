\chapter{Retrospektiva} \label{retrospective}

Tato bakalářská práce byla z mého pohledu velmi přínosná. Nicméně stejně jako každý vývoj se také tento potýkal s nejen pár problémy, a to nejen implementačními, ale i určitými rozpory s ohledem na preference a technologická rozhodnutí. V této kapitole budou tyto překážky popsány společně se způsoby, jakými byly řešeny.

\section{Technologické problémy}

Jeden z více osobních problémů, které nastaly, byla brzká technologická rozhodnutí. Důvody volby těchto technologií byly popsány v kapitole Využité technologie. Zde jsou popsány konkrétní překážky, jež byly spojovány s těmito rozhodnutími.

\subsection{Javascript}
Jak jsem již na začátku zmiňoval, tato technologická volba nese i značné nevýhody, jako například:

\begin{description}
\item[Dynamické datové typy] 
Dynamicky typované jazyky jsou náchylné na větší chybovost a menší kvalitu kódu \cite{pang2018programming}. V bakalářské práci se osvědčilo použití statických typů, které pomohly lépe definovat výstupy a vstupy.
\item[Chybějící vlastnosti jazyka] 
Jelikož JavaScript je dynamicky typovaný jazyk, chybí v něm koncepty jako například enum, interface a mnoho dalších, které například TS dodává \cite{typescriptlangHandbookEnums, typescriptlangHandbookInterfaces}.
\item[Řešení výjimek]
JS má oproti více moderním programovacím jazykům mně osobně nevyhovující systém odchytávání chyb. JS používá systém try/catch/finally, který se v moderních jazycích postupně nahrazuje konceptem Errors as values. Tato nevýhoda je pouze osobní preferencí, ale~tato preference je podmíněna. Pokud například mám určitou funkci, při jejímž průběhu může nastat chyba, tak v jazycích, které mají chybovatelné typy (například Result enum v Rustu), či funkce s více vracejícími hodnotami (jako například v Go), se jednoduše vrátí jiný výsledek. Díky tomuto přístupu je programátor nucen chybu ošetřit, aby se ujistil o existenci správného výsledku. To vede k robustnějšímu kódu. Pokud ale můžeme jednoduše výjimku vyhodit, čímž průběh programu narušíme a funkce stále vrací stejný typ, není přímo z definice této funkce jasné, zda může vyhodit výjimku. Jsme v tu chvíli nuceni tuto záležitost odchytávat při tom, jak na ni narazíme. To se stává problémem, když interagujeme s námi nenapsaným kódem.
\end{description}

Po několika měsících vývoje nastalo důležité technologické rozhodnutí ohledně přepsání celého backendu do jazyka TS. Toto rozhodnutí bylo klíčové pro další vývoj, jelikož vyřešilo dva ze tří problémů, které jsem s JavaScriptem měl.

TS a JS mají zároveň i jeden další problém, a tím je paralelismus se sdílenou pamětí. Na malé škále, jako je tato bakalářská práce, která nepocítila velké uživatelské zatížení, to není zřejmé, ale~myslím si, že by se aplikace lépe škálovala, kdybych backend napsal v jiném jazyce, jako je například Go.

\subsection{Implementace vlastního ORM}
I přesto, že jsem s momentální implementací ORM spokojený, zabrala velkou část vývoje. Tento čas by mohl být investován do jiných aspektů aplikace, které jsou popsány v možných rozšířeních práce. Tato časová investice nepřinesla až takové výhody, jaké jsem očekával. Bohužel na všechna rozšíření, jako programatické skládání dotazů, podmínek a migrační systém, čas nezbyl.
