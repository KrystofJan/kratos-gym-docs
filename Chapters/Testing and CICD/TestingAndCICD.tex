\chapter{Testy a CICD}
V softwarovém vývoji se lze potkat výhradně s dvěma typy chyb. Jedná se o chyby syntaktické a chyby logické. Syntaktické chyby odchytává compiler, či interpreter při kompilování, nebo běhu programu. Jedná se o takovou syntaktickou analýzu. V dnešní době se tyto chyby dají odchytnout i dříve pomocí moderních editorů, jež využívají LSP. Syntaktické chyby jsou tedy vcelku jednoduché na odchycení a lze se jich zbavit okamžitě. Logické chyby jsou v tomto ohledu mnohem komplikovanější. Namísto pomyslného boje proti compileru, nebo typové kontrole se jedná o boj proti sobě. Logické chyby se dají odchytit pouze užíváním výsledného produktu a zjištěním jiného stavu než je očekávaný. 

Řešení tohoto problému je jasné, po implementaci je potřeba program vyzkoušet a zjistit, zda všechny potřebné komponenty fungují tak, jak mají. Tento proces je ale velmi zdlouhavý a často opomenutý z několika důvodů, mezi které se zahrnuje jak lenost, tak vytíženost, stejně jako změna priorit. Pro zjednodušení a zrychlení celého tohoto procesu byla na bakalářské práci využita implementace testování a CICD.

\section{Testování}
Testy jsou pouze implementovány pro backendovou část bakalářské práce. Zpracování je díky pečlivému odchycení chyb a vracení správných Status kódů vymezeno pouze na poslání dotazu na API a následné kontrole daného HTTP kódu. Testy jsou implementovány v programovacím jazyce python s pomocí knihovny pytest.

\section{CI/CD}
CI/CD představuje sérii kroků provedených automaticky při změnách v kódu. V této práci byla tato vývojová filozofie adaptována v obou formách pomocí github actions. viz Využité technologie.

\section{CD - Continuous deployment}
Co se kontinuálního nasazení kódu týče, tak řeší tvorbu Docker kontejneru. Tento kontejner je vytvořen jinak pro jiné účely. Pokud je pud pro spuštění automatického nasazení vyvolán pomocí vytvoření/přidání změn do pull requestu, tak se vytvoří a pošle tag s odpovídajícím číslem pull requestu. Pokud se jedná o sloučení do hlavní vývojové větve main, tak se jedná o tag latest.

\section{CI - Continuous integration}
V rámci kontinuální integrace se řeší spuštění testů pro API. Obecně jde o proces zajištění splnění funkčních i nefunkčních parametrů, jež program aplikuje. Mezi funkční parametry se zahrnuje samotná funkčnost kódu a splnění nároků implementované funkce. Nefunkční parametry zahrnují věci obecné ke kvalitě kódu, linting, formátování, a pod.
