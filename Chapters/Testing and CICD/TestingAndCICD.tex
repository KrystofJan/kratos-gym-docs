\chapter{Testy a CI/CD} \label{testing&cicd}
Ve vývoji softwaru se často setkáváme s různými druhy chyb, včetně těch logických \cite{alzahrani2021common}. Jiné chyby, jako jsou syntaktické, mohou být detekovány během kompilace. V současnosti mohou být tyto chyby identifikovány i dříve prostřednictvím moderních editorů podporujících LSP. Syntaktické chyby jsou tak relativně snadno zjistitelné a okamžitě řešitelné. Naopak, logické chyby představují složitější problém. Zatímco u syntaktických chyb jde o interakci s kompilátorem nebo typovým systémem, logické chyby vyžadují introspektivní přístup. Tyto chyby lze detekovat výhradně při práci s konečným produktem, kdy je zpozorován rozdíl mezi očekávaným a skutečným stavem.

Řešení tohoto problému je jasné, po implementaci je potřeba program vyzkoušet a zjistit, zda všechny potřebné komponenty fungují tak, jak mají. Tento proces je ale velmi zdlouhavý a často opomenutý z několika důvodů, mezi které se zahrnuje jak pohodlnost, tak vytíženost, stejně jako změna priorit. Pro zjednodušení a zrychlení celého tohoto procesu byla na bakalářské práci využita implementace testování a CI/CD.

\section{Testování}
Testy jsou implementovány pouze pro backendovou část bakalářské práce. Zpracování je díky pečlivému odchycení chyb a vracení správných status kódů vymezeno pouze na poslání dotazu na API a~následné kontrole daného HTTP kódu. Testy jsou implementovány v programovacím jazyce \texttt{Python} s pomocí knihovny \texttt{pytest}.

\section{CI/CD}
CI/CD představuje sérii kroků provedených automaticky při změnách v kódu. V této práci byla tato vývojová filozofie adaptována v obou formách pomocí github actions. viz \ref{techstack} Využité technologie. Všechny tyto technologie jsou úzce spojené s verzovacími koncepty, jako je koncept různých vývojových větví, pull requestů a pod. Tyto koncepty jsou specifické pro zvolený verzovací systém git a cloudové uložiště GitHub. Celá praktická část této bakalářské práce je uložena na GitHubu \url{https://github.com/KrystofJan/kratos-gym}

\subsection{CD - Continuous deployment}
Co se kontinuálního nasazení kódu týče, tak řeší tvorbu Docker kontejneru. Tento kontejner je vytvořen jinak pro jiné účely. Když je automatické nasazení iniciováno pomocí vytvoření nebo~úprav pull requestu, je generován a odeslán tag, který odpovídá identifikačnímu číslu daného pull requestu. Pokud se jedná o sloučení do hlavní vývojové větve main, tak se jedná o tag latest.

\subsection{CI - Continuous integration}
V rámci kontinuální integrace se řeší spuštění testů pro API. Obecně jde o proces zajištění splnění funkčních i nefunkčních parametrů, jež program aplikuje. Mezi funkční parametry se zahrnuje samotná funkčnost kódu a splnění nároků implementované funkce. Nefunkční parametry zahrnují věci obecné ke kvalitě kódu, linting, formátování, a pod.

\section{Git \& Github}
Jelikož jsou všechny tyto koncepty spojené s verzováním kódu, je potřeba zmínit, že pro aplikace těchto postupů a verzování kódu byly využity technologie Git a GitHub. Pro bližší popis těchto technologií se lze odkázat do kapitoly \ref{techstack}.
