
\subsection{Struktura}
Struktura projektu prošla několika výraznými změnami. S každou iterací systém nabýval na komplexitě a více strádal na přehlednosti, což vedlo k rozhodnutí provést kompletní přepsání backendu. Toto rozhodnutí vedlo k výraznému zlepšení přehlednosti a jednoduchosti systému. Podrobné poznámky k procesu refactoringu a jednotlivým iteracím jsou uvedeny v následujících částech textu.

Aktuální backendová architektura je koncipována jako přehledně strukturované řešení s explicitním důrazem na logické uspořádání komponent. Její organizace, založená na seskupování zdrojů dle koncových bodů (např. rezervace, uživatelský účet, atd.), místo tradičního rozdělení podle funkcí (např. service, controller, atd.)
Kořenový adresář je uspořádán následujícím způsobem:
\dirtree{%
.1 db/.
.2 create\_script.sql.
.1 tests/.
.2 ....
.1 src/.
.2 database/.
.3 decorators/.
.4 database-decorators.ts.
.3 basic-query-database.ts.
.3 database-response.ts.
.3 database.ts.
.2 endpoints/.
.3 reservation/.
.4 reservation.controller.ts.
.4 reservation.database.ts.
.4 reservation.error-handler.ts.
.4 reservation.model.ts.
.4 reservation.routes.ts.
.4 reservation.service.ts.
.3 ....
.2 errors/.
.3 base.error.ts.
.3 error.handler.ts.
.2 request-utility/.
.3 common/.
.4 response-status.ts.
.3 custom-responses/.
.4 ok-response.ts.
.4 created-response.ts.
.4 ....
.2 router/.
.3 router.ts.
.2 utils/.
.3 ....
.2 app.ts.
.2 package.json.
.2 package-lock.json.
}
\subsection{Root složka}
\begin{description}
    \item[db] 
    Zde se ukládají SQL scripty pro vytvoření databáze, vkládání testovacích dat další podobné scripty.
    \item[tests]  
    Zde jsou uloženy testy napsány v pythonu pomocí knihovny pytest. Stručný popis testování můžete nalést v kapitole "Testy a CICD"
    \item[src] 
    V této složce se nachází veškerá logika aplikace, která je podrobně popsána níže.
\end{description}

\subsection{Složka src}
Složka src centralizuje veškerou logiku Express.js HTTP serveru a je strukturována do následujících klíčových modulů:
\begin{description}
    \item[database] 
    Složka database obsahuje logiku specifickou pro ORM a obecné databázové operace. Jsou zde definovány nástroje, jako například výše zmíněné dekorátory pro mapování dat. Nachází se zde také typy reprezentující odpovědi z databáze, což zjednodušuje práci při manipulaci s daty.
    \item[endpoints]
    Každý koncový bod v projektu obsahuje následující soubory: controller, error-handler, model, routes a service. V některých případech může obsahovat i část database, která je zodpovědná za specifické databázové operace, jako jsou volání procedur nebo složité transakce.
    \item[errors] 
    Složka errors obsahuje základní error handler, ze kterého vycházejí ostatní handlery. Dále obsahuje definici typu CodedError, což je vlastní typ chyby obsahující kód chyby a odpovídající hlášku. Součástí je také ErrorCode, což je výčet chyb, které mohou nastat.
    \item[request-utility] 
    Složka request-utility obsahuje definice těl HTTP odpovědí, které se vrací z controllerů. Patří sem například:
    CreatedResponse pro úspěšné POST požadavky.
    DeletedResponse pro úspěšné DELETE požadavky.
    \item[router] 
    Router definuje všechny HTTP cesty, které API nabízí, a volá specifické routery pro jednotlivé endpointy.
    \item[utils] 
    Složka utils obsahuje pomocné funkce a typy, které usnadňují práci. Je zde například typ IDictionary (typová definice mapy pro libovolné datové typy), funkce safeAwait (funkce pro zjednodušení práce s try-catch. Umožňuje práci s chybami v podobném stylu jako v jazyce Go, což přispívá k přehlednosti kódu), definici loggeru (využívá PINO knihovnu) a pod.
    \item[app.ts] 
    Soubor app.ts slouží jako vstupní bod aplikace. Jeho hlavní funkcí je vytvoření instance frameworku ExpressTs
\end{description} 

\subsection{Endpoints / Koncové body}
V další části popíšeme strukturu modulů a na příkladech ukážeme tok dat od rout přes aplikační logiku k databázi a zpět ke klientovi:

\subsection{model}
Reprezentuje data vrácená v odpovědi. Model využívá dekorátory, které definují mapování mezi daty v modelu a databází. Jedná se o třídu, která dědí z obecného modelu.

\lstinputlisting[label=src:BERoutes,caption={Příklad implementace endpoint modelu}]{SourceCodes/foo.model.ts}

Tento příklad demonstruje definiční datový model Foo s následující strukturou:
\begin{description}
    \item[FooId (typ number)] automaticky generovaný primární klíč tabulky
    \item[Bar (typ Bar)] cizí klíč definující 1:1 asociaci s tabulkou Bar.
\end{description} 

Model zároveň ilustruje aplikaci dekorátorů (např. @PrimaryKey, @ForeignKey, @Column, atd.), které rozšiřují funkcionalitu modelu o metadata nutná pro obecné požadavky na databázi. 

Následující příklad bude popisován dále v popisech dalších bodech.
\subsection{routes}
Obsahuje definice HTTP cest, které se volají v globálním routeru. Tyto definice obvykle pouze směrují požadavky na příslušný controller.

\lstinputlisting[label=src:BERoutes,caption={Příklad implementace endpoint routů}]{SourceCodes/foo.routes.ts}

V tomto příkladu nejprve vytvoříme instanci express.js routeru. Následovně definujeme jednotlivé koncové body, které budou pro uživatele dostupné. Tyto definice pouze volají příslušný controller, kterému předává informace o requestu (dotazu) a o požadované response (odpovědi)

\subsection{controller} 
Zajišťuje logiku volání příslušných služeb (service) a sestavuje tělo HTTP odpovědi na základě výsledků těchto služeb.

\lstinputlisting[label=src:BEControllers,caption={Příklad implementace endpoint controllerů}]{SourceCodes/foo.controller.ts}

Kontroler FooController využívá safeAwait pro asynchronní error handling bez try/catch, inspirovaný Go jazykem. Metoda FindById nejprve načte entitu Foo, poté asociovaný Bar, a standardizuje odpovědi přes OkResponse/FailedResponse.

\subsection{service}
Úloha service je na základě vstupních dat vrátit data namapovaná na požadovaný typ. Tyto data je možné získat z databáze, nebo jiných služeb.

\lstinputlisting[label=src:BEServices,caption={Příklad implementace endpoint service}]{SourceCodes/foo.service.ts}

Příklad FooService ukazuje, jak metoda GetFooById využívá BasicQueryDatabase pro komunikaci s databází a safeAwait ke zpracování chyb bez nutnosti zanořených try/catch bloků. Surová data z dotazu jsou transformována do instance třídy Foo pomocí explicitního mapování, které zajišťuje validaci a typovou bezpečnost.

\subsection{database}
Pokud nejsou obecné operace pro práci v databázi pro potřebnou akci dostatečné, lze vytvořit instance databáze, která se od obecného řešení odvíjí, ale zároveň implementuje specifickou logiku vyžadovanou danými požadavky.

\lstinputlisting[label=src:BEServices,caption={Příklad implementace endpoint service}]{SourceCodes/bd-select.ts}

SelectSpecific je jedna z funkcí třídy BasicQueryDatabase, která využívá metadata navázaná na objekt pomocí dekorátorů. Jak lze vidět funkce se převážne jen stará o sestavení SQL dotazu, vrácení získaných dat a uzavření spojení s databází.

\subsection{error-handler}
Slouží jako mapper, který na základě vrácených chyb sestavuje odpovídající HTTP odpovědi. Pod pokličkou se jedná o hash mapu, která obsahuje typy errorů se status kódy. Je volán z controlleru.
