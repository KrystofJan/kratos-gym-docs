\chapter{Závěr}

\section{Získané zkušenosti}

Celkově si troufám říci, že jsem během tohoto procesu nabyl spoustu zkušeností, které jsem schopen aplikovat v praxi. Jsem nyní více sebevědomý ve svých schopnostech. Ustanovil jsem si určité preference, jako například větší oblibu striktně typovaných kompilovaných jazyků. Oprášil jsem si své frontendové dovednosti, jež jsem od studia střední školy mimo stylování nepoužil. Dále jsem se samozřejmě zlepšil v mých schopnostech psát JavaScript a TypeScript.

\section{Budoucí směřování a rozšíření}

Následující seznam obsahuje dodatky, které bych rád do práce přidal v budoucnu:

\begin{enumerate}
\item \textbf{Kreditový systém} – Každá rezervace by byla za určitý počet kreditů, tyto kredity by se odvíjely od skutečných peněz. Zákazník by mohl například dostávat i věrnostní kredity.
\item \textbf{Napojení platební služby} – Kreditový systém by potřeboval i platební službu, jako například Stripe, jež by umožnila zasílat platby přes internet.
\item \textbf{Statistiky} – Rád bych do administrace přidal statistiky chodu posilovny, například nejpoužívanější přístroje.
\item \textbf{Mapa posilovny} – Konfigurátor jsem původně chtěl koncipovat v podobě interaktivní mapy posilovny. Toto řešení bylo nakonec zaměněno za checkboxy a konfiguraci časových oken. Rád bych ale tento způsob prozkoumal.
\item \textbf{Uložení tréninkového plánu} – V databázi pro předpřipravené tréninkové plány již existují potřebné tabulky, ale k implementaci jsem se nedostal.
\item \textbf{Generace tréninku s "dírami"} – Jako díry si můžeme představit volný čas mezi časovými okny zvolenými pro přístroj. Generování těchto tréninků by mohlo teoreticky vést k efektivnějšímu využití přístrojů.
\end{enumerate}

Tato rozšíření by dále zvýšila praktickou hodnotu aplikace a poskytla uživatelům komplexnější řešení pro správu jejich tréninkových aktivit.
\endinput
